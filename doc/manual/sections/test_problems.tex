%===============================================================================
\section{Test Problems}
%===============================================================================
\subsection{Steady-State Diffusion Test Problem}
%-------------------------------------------------------------------------------
The following parameters were used:
\[
   x\in(0,3),
   \quad \sigma_a = 0.25,
   \quad \sigma_s = 0.75,
   \quad j^+(0) = j^-(3) = 0,
   \quad Q = 1
\]
The exact solution to this problem is the following:
\[
   \phi(x) = A\sinh\fn{\frac{x}{L}} + B\cosh\fn{\frac{x}{L}} + Q\frac{L^2}{D} \pec
\]
where $D=\frac{1}{3\sigma_t}$, $L=\sqrt{\frac{D}{\sigma_a}}$, and $A$ and $B$ are
complicated expressions that come out to the following values:
\[
   A=2.4084787907,\quad B=-2.7957606046 \pep
\]

\subsection{Steady-State Pure Absorber Test Problem}
%-------------------------------------------------------------------------------
The following parameters were used:
\begin{gather*}
   x\in(x_L,x_R)=(0,10),
   \quad \sigma_a = 0.1,
   \quad \sigma_s = 0,\\
   \psi^+(x_L) = 20,
   \quad \psi^-(x_R) = 30,
   \quad Q^\pm = 0
\end{gather*}
The exact solution to this problem is the following:
\[
   \psi^-(x) = \psi^-(x_R)e^{-\frac{\sigma_a}{\mu^-}\fn{x-x_R}} \pec
\]
\[
   \psi^+(x) = \psi^+(x_L)e^{-\frac{\sigma_a}{\mu^-}\fn{x-x_L}} \pep
\]

\subsection{Radiation Hydrodynamics MMS Problem}
%-------------------------------------------------------------------------------
To test the full radiation-hydrodynamics system, an MMS (Method of Manufactured
Solutions) problem was designed. The original system given by Equations
\eqref{eq:cons_mass} through \eqref{eq:cons_energy} and \eqref{eq:S2Q} is
the following:
\[
   \dxdy{\rho}{t}+\dxdy{}{x}\fn{\rho u} = 0 \pec
\] 
\[
   \dxdy{}{t}\fn{\rho u} + \dxdy{}{x}\fn{\rho u^2} + \dxdy{}{x}\fn{p}
     = \frac{\sigma_t}{c} \F_0 \pec
\]
\[
   \dxdy{E}{t} + \dxdy{}{x}\bracket{\fn{E+p}u}=-\sigma_a c \fn{aT^4 - \E}
     + \frac{\sigma_t u}{c} \F_0 \pec
\]
\[
  \frac{1}{c}\dydt{\Psi^\pm} + \mu^\pm\dydx{\Psi^\pm} + \sigma_t\Psi^\pm
  = \frac{\sigma_s}{2}\phi + \Q^\pm \pep
\]
To execute an MMS problem, extraneous source terms are added to the
system as follows:
\begin{subequations}
\begin{equation}
   \dxdy{\rho}{t}+\dxdy{}{x}\fn{\rho u} = Q^{ext,\rho} \pec
\end{equation} 
\begin{equation}
   \dxdy{}{t}\fn{\rho u} + \dxdy{}{x}\fn{\rho u^2} + \dxdy{}{x}\fn{p}
     = \frac{\sigma_t}{c} \F_0 + Q^{ext,u} \pec
\end{equation}
\begin{equation}
   \dxdy{E}{t} + \dxdy{}{x}\bracket{\fn{E+p}u}=-\sigma_a c \fn{aT^4 - \E}
     + \frac{\sigma_t u}{c} \F_0 + Q^{ext,E} \pec
\end{equation}
\begin{equation}
  \frac{1}{c}\dydt{\Psi^\pm} + \mu^\pm\dydx{\Psi^\pm} + \sigma_t\Psi^\pm
  = \frac{\sigma_s}{2}\phi + \Q^\pm + Q^{ext,\pm} \pep
\end{equation}
\end{subequations}
If one chooses functions for the velocity field $u(x,t)$
and any two independent, intensive thermodynamic properties, for example,
density $\rho(x,t)$ and total energy density $E(x,t)$, then one fully
defines the state of the fluid. Since the assumption that the homogeneous
solution for density $\rho^*(x,t^{n+1})$ is equal to the general solution
$\rho(x,t^{n+1})$ is used in implementation, it simplifies the execution
of the MMS problem to choose $\rho(x,t)$ and $u(x,t)$ such that the
MMS source $Q^{ext,\rho}(x,t)$ is zero, as this keeps the continuity
equation homogeneous and thus the assumption that $\rho^*=\rho$ remains valid.
This was achieved with the definitions
\begin{equation}
   \rho(x,t) = e^{x+t} \pec
\end{equation}
\begin{equation}
   u(x,t) = e^{-x}\sin t - 1 \pep
\end{equation}
To complete the definition of the fluid state, an arbitrary function of $(x,t)$
was chosen for total energy density:
\begin{equation}
   E = something
\end{equation}
