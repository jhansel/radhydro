\section{The Scheme}

The solution for a time step $t^n\rightarrow t^{n+1}$ consists of four
nonlinear solves:

\begin{enumerate}
  \item Crank-Nicolson step from $t^n$ to $t^{n+\fourth}$
  \item Crank-Nicolson step from $t^n$ to $t^{n+\half}$
  \item Crank-Nicolson step from $t^{n+\half}$ to $t^{n+\frac{3}{4}}$
  \item TR/BDF-2 step from $t^{n+\half}$ to $t^{n+1}$
\end{enumerate}

First, some notation is defined; the following are quantities that are stored
throughout the calculation:
\begin{center}
\begin{tabular}{l}
   $\R^n \equiv \{\Psi^\pm\iL,\Psi^\pm\iR\} \quad\forall i$ \\
   $\H^n \equiv \{\rho_i^n, (\rho u)_i^n, E_i^n\} \quad\forall i$ \\
   $\delta e^n \equiv \{\delta e_i^n\} \quad\forall i$ \\
   $\Delta^n \equiv \{\Delta\rho_i^n, \Delta(\rho u)_i^n, \Delta E_i^n\} \quad\forall i$ \\
   $\sigma^n \equiv \{\sigma_{s,i,L}^n, \sigma_{s,i,R}^n,
   \sigma_{a,i,L}^n, \sigma_{a,i,R}^n, \sigma_{t,i,L}^n, \sigma_{t,i,R}^n\}
   \quad\forall i$
\end{tabular}
\end{center}
Other quantities are computed when needed.

Now begins a description of the proposed scheme:

\begin{enumerate}

\item \textbf{Compute initial conditions.} The following initial conditions
are provided: $\R^0$, $\H^0$, $\sigma^0$, and $\delta e^0$. The internal
energy slopes $\delta e^0$ may just be initialized to zero.

\item \textbf{Begin transient.} For each time step $n$, a new time step
size must be computed in accordance with the CFL condition. Since two
full hydrodynamics steps are being taken per time step for the scheme,
the usual time step size computed from the CFL condition may be doubled:
\begin{equation}
   \dt \leq 2\frac{\dx_{min}}{c_{max}} \pec
\end{equation}
where $c_{max}^n$ is the maximum speed of sound computed in the domain
at time $t^n$.

% Cycle 1 predictor

\item \textbf{Perform Cycle 1 Predictor.} This step goes from
$t^n$ to $t^{n+\fourth}$.

\begin{enumerate}
\item \textbf{Perform MUSCL-Hancock Predictor.} First, slopes $\Delta^n$ are
computed via Equations \eqref{eq:muscl_slopes} and \eqref{eq:muscl_differences}.
Then a linear representation of the solution is created via
Equation \eqref{eq:edge_hydro}, and the
the MUSCL-Hancock predictor is performed to obtain $\H^*$, the
homogeneous hydrodyamics solution at $t^{n+\fourth}$,
via Equation \eqref{eq:muscl_predictor}.

\item \textbf{Initialize nonlinear iteration.} Iteration of the
$t^{n+\fourth}$ solution is required since the system of equations
is nonlinear. The iterates $\R^k$, $\H^k$, and $\sigma^k$ may be initialized
to the previous time
values, $\R^n$, $\H^n$, and $\sigma^n$ respectively.

\item\label{item:vel_update}
\textbf{Update velocities.} The Crank-Nicolson discretization
of the velocity update equation, Equation \requ{hydromCNfull},
is solved to obtain new
velocities at cell centers, $\{u_i^{k+1}\}$. Evaluation of
the cross sections and radiation quantities $\E$ and $\F$
at cell centers is achieved by averaging the left and right values.

\item \textbf{Update radiation.} The Crank-Nicolson discretization of the S-2
equations, Equations \requ{S2CNfullL} and \requ{S2CNfullR} are solved,
employing the linearization given in Section \ref{sec:linearization}.
Evaluation of the densities and velocities at the edges is achieved using using
the cell-centered values in conjunction with the slopes $\Delta^n$. Evaluation
of internal energies and temperatures is achieved using the internal energy
slopes $\delta e^n$.

\item \textbf{Update internal energies.} The internal energies are
updated in accordance with the linearization procedure given is Section
\ref{sec:linearization}. The update equations produce edge values
$\{e^{k+1}\iL,e^{k+1}\iR\}$. These left and right values are
averaged to produce cell average values for internal energy $\{e^{k+1}_i\}$
which are used in the subsequent iteration.
This is also the time when cross sections should be updated
if they are functions of the hydrodynamic state of the fluid.

\item \textbf{Check convergence.} The new solutions $\R^{k+1}$ and
$\H^{k+1}$ are compared with the previous iteration solutions
$\R^k$ and $\H^k$ to determine if convergence has been achieved.
If the solutions have not converged, then the computation
returns to Step \ref{item:vel_update}.
\end{enumerate}

% Cycle 1 corrector

\item \textbf{Perform Cycle 1 Corrector.} This step proceeds
just as the predictor step, except that the MUSCL-Hancock
corrector step given by Equation \eqref{eq:muscl_corrector} is used instead of the MUSCL-Hancock predictor
step, and the step goes from $t^n$ to $t^{n+\half}$ instead
of $t^n$ to $t^{n+\fourth}$. No new slopes are computed;
evaluation of edge densities and velocities again use
$\Delta^n$, and evaluation of edge internal energies
and temperatures again use $\delta e^n$. At the end
of the cycle, the new internal energy slopes $\delta e^{n+\half}$
are saved for the next cycle.

% Cycle 2 predictor

\item \textbf{Perform Cycle 2 Predictor.} This step proceeds
just as the Cycle 1 predictor step, except that the
step goes from $t^{n+\half}$ to $t^{n+\frac{3}{4}}$ instead
of $t^n$ to $t^{n+\fourth}$. As in Cycle 1, new slopes
are computed in the MUSCL-Hancock predictor step;
these slopes $\Delta^{n+\half}$ are then used for
evaluation of edge densities and velocities in the remainder
of the cycle. Evaluation of edge internal energies
and temperatures use the internal energies saved
from the end of Cycle 1, $\delta e^{n+\half}$.

% Cycle 2 corrector

\item \textbf{Perform Cycle 2 Corrector.} This step proceeds
as the Cycle 1 corrector step, except that the time step goes
from $t^{n+\half}$ to $t^{n+1}$, and the time discretization
of the equations is a form of TR/BDF-2
instead of Crank-Nicolson, so values at $t^n$
are used in the temporal discretization. Slopes
$\Delta^{n+\half}$ and $\delta e^{n+\half}$ are again
used to evaluate edge quantities. At the end
of the cycle, the new internal energy slopes $\delta e^{n+1}$
are saved for the next cycle.

% End

\item \textbf{Store values for next time step.}
At this point, the old solutions, hydrodynamics slopes,
internal energy slopes, and cross sections
are saved for the next time step.

\end{enumerate}

