%===============================================================================
\section{Results}
%===============================================================================
\subsection{Radiation Hydrodynamics MMS Problem}
%-------------------------------------------------------------------------------
To test the full radiation-hydrodynamics system, an MMS (Method of Manufactured
Solutions) problem was designed. The original system given by Equations
\eqref{eq:cons_mass} through \eqref{eq:cons_energy} and \eqref{eq:S2Q} is
the following:
\[
   \dxdy{\rho}{t}+\dxdy{}{x}\fn{\rho u} = 0 \pec
\] 
\[
   \dxdy{}{t}\fn{\rho u} + \dxdy{}{x}\fn{\rho u^2} + \dxdy{}{x}\fn{p}
     = \frac{\sigma_t}{c} \F_0 \pec
\]
\[
   \dxdy{E}{t} + \dxdy{}{x}\bracket{\fn{E+p}u}=-\sigma_a c \fn{aT^4 - \E}
     + \frac{\sigma_t u}{c} \F_0 \pec
\]
\[
  \frac{1}{c}\dydt{\Psi^\pm} + \mu^\pm\dydx{\Psi^\pm} + \sigma_t\Psi^\pm
  = \frac{\sigma_s}{2}\phi + \Q^\pm \pep
\]
To execute an MMS problem, extraneous source terms are added to the
system as follows:
\begin{subequations}
\begin{equation}
   \dxdy{\rho}{t}+\dxdy{}{x}\fn{\rho u} = Q^{ext,\rho} \pec
\end{equation} 
\begin{equation}
   \dxdy{}{t}\fn{\rho u} + \dxdy{}{x}\fn{\rho u^2} + \dxdy{}{x}\fn{p}
     = \frac{\sigma_t}{c} \F_0 + Q^{ext,\rho u} \pec
\end{equation}
\begin{equation}
   \dxdy{E}{t} + \dxdy{}{x}\bracket{\fn{E+p}u}=-\sigma_a c \fn{aT^4 - \E}
     + \frac{\sigma_t u}{c} \F_0 + Q^{ext,E} \pec
\end{equation}
\begin{equation}
  \frac{1}{c}\dydt{\Psi^\pm} + \mu^\pm\dydx{\Psi^\pm} + \sigma_t\Psi^\pm
  = \frac{\sigma_s}{2}\phi + \Q^\pm + Q^{ext,\pm} \pep
\end{equation}
\end{subequations}
Two MMS problems, one for the streaming limit and one for the equilibrium
diffusion limit, were taken from \cite{mcclarren2}.

For the diffusion limit problem, the following solutions were prescribed
for the hydrodynamic solution:
\begin{subequations}
\begin{equation}
  \rho = A\left(\sin(Bx - Ct) + 2\right) \pec
\end{equation}
\begin{equation}
  u = A\left(\cos(Bx - Ct) + 2\right) \pec
\end{equation}
\begin{equation}
  p = A\alpha\left(\cos(Bx - Ct) + 2\right) \pec
\end{equation}
\end{subequations}
where $A$, $B$, and $C$ are constants to be chosen.
As stated in \cite{mcclarren2}, the radiation energy density and flux
are the following in the equilibrium diffusion limit:
\begin{subequations}
\begin{equation}
  \E = T^4 \pec
\end{equation}
\begin{equation}
  \F = \frac{1}{3\sigma}\partial_x T^4 + \frac{4}{3}\frac{u}{\mathbb{C}}T^4 \pec
\end{equation}
\end{subequations}
where $\mathbb{C}$ is a non-dimensional parameter.
Using the ideal gas equation of state relations, an expression
for temperature $T$ can be derived and substituted into these equations,
using a specific heat $c_v$ such that
\begin{equation}
  T = \frac{\gamma p}{\rho} \pep
\end{equation}
Symbolic manipulation can be used to derive the MMS sources
$Q^\rho$, $Q^{\rho u}$, $Q^E$, and $Q^{\pm}$; the resulting expressions
are omitted here due to their complexity and length. The values chosen
for the constants/non-dimensional parameters are $A=B=C=1$, $\alpha=0.5$,
$\gamma=\frac{5}{3}$, and $\mathbb{C}=\sigma=1000$.
Figure \ref{fig:diffusion_limit_solution} shows the results for this
problem.

%\begin{figure}[ht]
%   \centering
%   \includegraphics[width=\textwidth]{diffusion_limit_solution.pdf}
%   \caption{}
%   \label{fig:diffusion_limit_solution}
%\end{figure}

\subsection{Radiation-Hydrodyamics Shocks}
%-------------------------------------------------------------------------------
blah blah \cite{edwardsthesis}.
