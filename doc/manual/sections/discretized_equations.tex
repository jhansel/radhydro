%===============================================================================
\section{The Discretized Equations}\lsec{full}
%===============================================================================

%===============================================================================
\section{MUSCL-Hancock Algorithm}
%===============================================================================

The homogeneous Euler equations may be expressed in conservative form as
\begin{equation}
  \dydt{\H} + \nabla\cdot\Flux(\H) = \mathbf{0} \pec
\end{equation}
where $\H$ is a vector of the conservative unknowns
and $\Flux(\H)$ is the flux associated with each:
\begin{equation}
  \H=\left[\begin{array}{c}\rho\\\rho u\\E\end{array}\right] \pec\qquad
  \Flux(\H)=\left[\begin{array}{c}\rho u\\
  \rho u^2 + p\\
  (E+p)u\end{array}\right] \pep
\end{equation}
The first half of a MUSCL-Hancock step $t^n\rightarrow t^n+\dt$
consists of constructing a linear representation of the solution
using slopes $\Delta_i^n$:
\begin{subequations}\label{eq:hydro_edge}
\begin{equation}
  \H\iL^n = \H_i^n - \frac{\Delta_i^n}{2} \pec
  \quad
  \H\iR^n = \H_i^n + \frac{\Delta_i^n}{2} \pec
\end{equation}
\begin{equation}\label{eq:slopes}
  \Delta_i^n = \half\Delta\H_{i-\half}^n + \half\Delta\H_{i+\half}^n \pec
\end{equation}
\begin{equation}
  \Delta\H_{i-\half}^n = \H_i^n - \H_{i-1}^n \pec\quad
  \Delta\H_{i+\half}^n = \H_{i+1}^n - \H_i^n \pec
\end{equation}
\end{subequations}
and then evolving that representation by half a time step:
\hydroPredictor{n}{n+\half}{\half}
The second half of a MUSCL-Hancock step employs a Riemann solver
to compute the fluxes at time $n+\half$:
\hydroCorrector{n}{n+\half}{n+1}{}


%-------------------------------------------------------------------------------
\subsection{Fluid Momentum Source Update Equation}
%-------------------------------------------------------------------------------
The Crank-Nicolson discretization of the velocity update equation is the
following:

\momentumUpdateCN{n}{}{_i}{\lequ{hydromCNfull}}

The BDF2 discretization of the velocity update equation is the
following:

\momentumUpdateBDFTwo{n-1}{n}{}{_i}{\lequ{hydromBDF2full}}

The momentum source terms in these equations are computed by averaging the
momentum source expression evaluated at the edges:

\begin{multline}
   \left[\frac{\sigma_t}{c}\left(\F - \frac{4}{3}\E u\right)\right]_i =
   \half\left[\frac{\sigma_{t,i,L}}{c}\left(\F\iL - \frac{4}{3}\E\iL u\iL\right)\right]\\
   + \half\left[\frac{\sigma_{t,i,R}}{c}\left(\F\iR - \frac{4}{3}\E\iR u\iR\right)\right]
   \pep
\end{multline}

%-------------------------------------------------------------------------------
\subsection{Fluid Energy Source Update Equation}
%-------------------------------------------------------------------------------
The energy source update equations are evaluated at the edges and thus hydrodynamic
quantities need to be evaluated at the edges. The stored solution values
consist of the cell average unknowns for the conservative variables $[\rho,\rho u,E]$
and their slopes $\Delta$, which come from the predictor step of MUSCL-Hancock,
as given by Equation \eqref{eq:muscl_slopes}. Evaluation of edge densities is achieved
by applying the slopes as given by Equation \eqref{eq:edge_hydro}:
\begin{equation}
   \rho\iL^k = \rho_i^k - \frac{\Delta\rho_i^n}{2} \pep
\end{equation}
Note that the slopes $\Delta$ are only updated in the MUSCL-Hancock predictor
step. Thus, quantities at $t^{n+1}$ also use the $\Delta^n$ slopes. For the
BDF2 discretization, where evaluation of quantities at $t^{n-1}$ is required,
the old slopes $\Delta^{n-1}$ are used. Evaluation of edge velocities is
achieved by evaluating the edge conservative variables $\rho$ and $\rho u$
and then computing velocity from these:
\begin{equation}
   u\iL^k = \frac{(\rho u)\iL^k}{\rho\iL^k}
          = \frac{(\rho u)_i^k - \frac{\Delta(\rho u)_i^n}{2}}
                 {\rho_i^k - \frac{\Delta\rho_i^n}{2}} \pep
\end{equation}
As explained in Section \ref{diffusion_limit}, internal energies (as well
as temperatures) use slopes that are independent of the MUSCL-Hancock slopes.
These internal energy slopes are denoted by $\delta e$. Evaluation
of edge internal energies is thus performed as follows:
\begin{equation}
   e\iL^k = e_i^k - \frac{\delta e_i^n}{2} \pep
\end{equation}
Edge temperatures are computed from the edge internal energies:
\begin{equation}
   T\iL^k = \frac{e\iL^k}{c_v} \pep
\end{equation}
Again, quantities at $t^{n+1}$ still use old slopes $\delta e^n$. These
slopes are updated at the end of each full MUSCL-Hancock step, i.e.,
when the nonlinear iteration for the corrector step converges, the
slopes $\delta e_i^{n+1}$ are computed from the converged edge internal energies:
\begin{equation}
   \delta e_i^{n+1} = e\iR^{k+1} - e\iL^{k+1} \pep
\end{equation}

The Crank-Nicolson discretization of the energy update equation is the following:

\energyUpdateCN{n}{}{\iL}{\lequ{hydroECNfull}}

The right edge equations are identical in form, obtained by replacing ``$L$'' with
``$R$'' in the left edge equations.
The BDF2 discretization of the energy update equation is the following:

\energyUpdateBDFTwo{n-1}{n}{}{\iL}{\lequ{hydroEBDF2full}}

Note that these energy update equations are not in the form in which they
are actually solved in practice; the implicit term $\sigma_a^k a c (T^{k+1})^4$
must be linearized. This procedure is detailed in Section \ref{sec:linearization}.
The final form of the energy update equation is a direct update to the
edge internal energies $\{e\iL^{k+1},e\iR^{k+1}\}$.

%-------------------------------------------------------------------------------
\subsection{\texorpdfstring{S$_2$}{S-2} Equations}
%-------------------------------------------------------------------------------

A lumped linear discontinuous (LLD) spatial discretization is employed
for the S$_2$ equations, so the angular flux
unknowns are the left and right values $\Psi_{i,L}^\pm$ and
$\Psi_{i,R}^\pm$ for each cell $i$. The spatially
discretized equations result from integrating each half cell
$(x_i-\frac{h_i}{2},x_i)$ and $(x_i,x_i+\frac{h_i}{2})$,
where $x_i$ is the cell center, $h_i$ is the cell width,
and the cell average and edge fluxes are defined as
\begin{equation}
  \Psi_i^\pm=\half\fn{\Psi_{i,L}^\pm + \Psi_{i,R}^\pm} \pec
\end{equation}
\begin{equation}
  \Psi_{i+\half}^+=\Psi_{i,R}^+ \pec\qquad \Psi_{i+\half}^-=\Psi_{i+1,L}^- \pep
\end{equation}

% "if" block for developer mode
\iftoggle{DEVELOPERMODE}{
%===============================================================================
\subsubsection{Steady-State}\lsec{S2ssdisc}
%===============================================================================
Using an LLD discretization, the steady-state S$_2$ equations, obtained by
dropping the $\dydt{\Psi^\pm}$ term in Equation \requ{S2Q}, become
\begin{equation}\lequ{S2ssL}
  \frac{2\mu^\pm}{h_i}\fn{\Psi^{\pm,k+1}_i - \Psi^{\pm,k+1}_{i-\half}}
  +\sigma_{t,i,L}^k\Psi\iL^{\pm,k+1}
  -\frac{\sigma_{s,i,L}^k}{2}\phi\iL^{k+1}
  =\Q_{i,L}^{\pm,k} \pec
\end{equation}
\begin{equation}\lequ{S2ssR}
  \frac{2\mu^\pm}{h_i}\fn{\Psi^{\pm,k+1}_{i+\half} - \Psi^{\pm,k+1}_i}
  +\sigma_{t,i,R}^k\Psi\iR^{\pm,k+1}
  -\frac{\sigma_{s,i,R}^k}{2}\phi\iR^{k+1}
  =\Q_{i,R}^{\pm,k} \pep
\end{equation}

Coefficients of the resulting linear system matrix are the following,
where conditions in parentheses indicate when a term does not
apply for the boundary elements $i=1$ and $i=N$:
\begin{center}
\begin{tabular}{|l||c|c|c|}\hline
          & $i-1,R,+$ & $i,L,-$ & $i,L,+$\\\hline\hline
  $i,L,-$ &
          & $-\frac{\mu^-}{h_i} + \sigtL - \sigsL$
          & $-\sigsL$ \\\hline
  $i,L,+$ & $-\frac{2\mu^+}{h_i}$ ($i\ne 1$)
          & $-\sigsL$
          & $\frac{\mu^+}{h_i} + \sigtL - \sigsL$ \\\hline
  $i,R,-$ &
          & $-\frac{\mu^-}{h_i}$
          & \\\hline
  $i,R,+$ &
          &
          & $-\frac{\mu^+}{h_i}$ \\\hline
\end{tabular}
\end{center}
\begin{center}
\begin{tabular}{|l||c|c|c|}\hline
          & $i,R,-$ & $i,R,+$ & $i+1,L,-$\\\hline\hline
  $i,L,-$ & $\frac{\mu^-}{h_i}$
          &
          & \\\hline
  $i,L,+$ &
          & $\frac{\mu^+}{h_i}$
          & \\\hline
  $i,R,-$ & $-\frac{\mu^-}{h_i} + \sigtR - \sigsR$
          & $-\sigsR$
          & $\frac{2\mu^-}{h_i}$ ($i\ne N$) \\\hline
  $i,R,+$ & $-\sigsR$
          & $\frac{\mu^+}{h_i} + \sigtR - \sigsR$
          & \\\hline
\end{tabular}
\end{center}
The right hand side vector for the linear system is the following,
where the columns indicate to which cell(s) the expressions
correspond:

\begin{center}
\begin{tabular}{|l||c|c|c|}\hline
          & $i=1$ & $i=2\ldots N-1$ & $i=N$\\\hline\hline
  $i,L,-$ & $\Q^-_{i,L}$
          & $\Q^-_{i,L}$
          & $\Q^-_{i,L}$ \\\hline
  $i,L,+$ & $\Q^+_{i,L} + \frac{2\mu^+}{h_i}\Psi^+_{inc}$
          & $\Q^+_{i,L}$
          & $\Q^+_{i,L}$ \\\hline
  $i,R,-$ & $\Q^-_{i,R}$
          & $\Q^-_{i,R}$
          & $\Q^-_{i,R} - \frac{2\mu^-}{h_i}\Psi^-_{inc}$ \\\hline
  $i,R,+$ & $\Q^+_{i,R}$
          & $\Q^+_{i,R}$
          & $\Q^+_{i,R}$ \\\hline
\end{tabular}
\end{center}

The incoming boundary fluxes $\Psi^\pm_{inc}$ are computed from the
user-supplied boundary half-range currents $j^\pm$ using quadrature:
\begin{equation}
   j^+(x_L) = 2\pi\int\limits_0^1 \psi(x_L,\mu)\mu d\mu
   \approx \sum\limits_{\mu_d>0}\psi(x_L,\mu_d)\mu_d w_d
   = \mu^+\Psi^+_{inc} \pec
\end{equation}
\begin{equation}
   j^-(x_R) = -2\pi\int\limits_{-1}^0 \psi(x_R,\mu)\mu d\mu
   \approx -\sum\limits_{\mu_d<0}\psi(x_R,\mu_d)\mu_d w_d
   = -\mu^-\Psi^-_{inc} \pep
\end{equation}

Transient S$_2$ solvers can employ a steady-state solver if
the following generalization is made to Equations \requ{S2ssL}
and \requ{S2ssR}:
\begin{equation}\lequ{S2trL}
  \alpha\frac{2\mu^\pm}{h_i}\fn{\Psi^{\pm,k+1}_i - \Psi^{\pm,k+1}_{i-\half}}
  +\fn{\alpha\sigma_{t,i,L}^k + \tau}\Psi\iL^{\pm,k+1}
  -\alpha\frac{\sigma_{s,i,L}^k}{2}\phi\iL^{k+1}
  =\tilde{\Q}_{i,L}^{\pm,k} \pec
\end{equation}
\begin{equation}\lequ{S2trR}
  \alpha\frac{2\mu^\pm}{h_i}\fn{\Psi^{\pm,k+1}_{i+\half} - \Psi^{\pm,k+1}_i}
  +\fn{\alpha\sigma_{t,i,R}^k + \tau}\Psi\iR^{\pm,k+1}
  -\alpha\frac{\sigma_{s,i,R}^k}{2}\phi\iR^{k+1}
  =\tilde{\Q}_{i,R}^{\pm,k} \pec
\end{equation}
where $\alpha$, $\tau$, and $\tilde{\Q}^\pm$ depend on the time
discretization used.


}{} % end of "if" block for developer mode; second pair of brackets is the "else" block

\subsubsection{Crank-Nicolson}\lsec{S2fullCN}
%-------------------------------------------------------------------------------
\begin{equation}\lequ{S2CNfullL}\begin{split}
  \frac{1}{c}\frac{\Psi\iL^{\pm,k+1}-\Psi\iL^{\pm,n}}{\dt} = &
  -\half\frac{2\mu^\pm}{h_i}\fn{\Psi^{\pm,n}_i - \Psi^{\pm,n}_{i-\half}}
  -\half\frac{2\mu^\pm}{h_i}\fn{\Psi^{\pm,k+1}_i - \Psi^{\pm,k+1}_{i-\half}}\\
  &-\half\sigma_{t,i,L}^n\Psi\iL^{\pm,n}
   -\half\sigma_{t,i,L}^k\Psi\iL^{\pm,k+1}\\
  &+\half\frac{\sigma_{s,i,L}^n}{2}\phi\iL^n
   +\half\frac{\sigma_{s,i,L}^k}{2}\phi\iL^{k+1}\\
  &+\half\Q_{i,L}^{\pm,n}
   +\half\Q_{i,L}^{\pm,k} \pep
\end{split}\end{equation}

\begin{equation}\lequ{S2CNfullR}\begin{split}
  \frac{1}{c}\frac{\Psi\iR^{\pm,k+1}-\Psi\iR^{\pm,n}}{\dt} = &
  -\half\frac{2\mu^\pm}{h_i}\fn{\Psi^{\pm,n}_{i+\half} - \Psi^{\pm,n}_i}
  -\half\frac{2\mu^\pm}{h_i}\fn{\Psi^{\pm,k+1}_{i+\half} - \Psi^{\pm,k+1}_i}\\
  &-\half\sigma_{t,i,R}^n\Psi\iR^{\pm,n}
   -\half\sigma_{t,i,R}^k\Psi\iR^{\pm,k+1}\\
  &+\half\frac{\sigma_{s,i,R}^n}{2}\phi\iR^n
   +\half\frac{\sigma_{s,i,R}^k}{2}\phi\iR^{k+1}\\
  &+\half\Q_{i,R}^{\pm,n}
   +\half\Q_{i,R}^{\pm,k} \pep
\end{split}\end{equation}

% "if" block for developer mode
\iftoggle{DEVELOPERMODE}{

These equations can be solved by solving steady-state S$_2$ equations
of the form of Equations \requ{S2trL} and \requ{S2trR} by making
the following definitions:
\begin{equation}
  \tau = \frac{1}{c\dt} \pec \quad \alpha = \frac{1}{2}\pec
\end{equation}
\begin{multline}
  \tilde{\Q}_{i,L}^{\pm,k}
  = \frac{\Psi\iL^{\pm,n}}{c\dt}
  - \half\frac{2\mu^\pm}{h_i}\fn{\Psi_i^{\pm,n} - \Psi_{i-\half}^{\pm,n}}
  - \half\sigma_{t,i,L}^n\Psi\iL^{\pm,n}\\
  + \half\frac{\sigma_{s,i,L}^n}{2}\phi\iL^n
  + \half\Q\iL^{\pm,n}
  + \half\Q\iL^{\pm,k}
\end{multline}
\begin{multline}
  \tilde{\Q}_{i,R}^{\pm,k}
  = \frac{\Psi\iR^{\pm,n}}{c\dt}
  - \half\frac{2\mu^\pm}{h_i}\fn{\Psi_{i+\half}^{\pm,n} - \Psi_i^{\pm,n}}
  - \half\sigma_{t,i,R}^n\Psi\iR^{\pm,n}\\
  + \half\frac{\sigma_{s,i,R}^n}{2}\phi\iR^n
  + \half\Q\iR^{\pm,n}
  + \half\Q\iR^{\pm,k}
\end{multline}

}{} % end of "if" block for developer mode; second pair of brackets is the "else" block

\subsubsection{TR/BDF-2}\lsec{S2fullTRBDF2}
%-------------------------------------------------------------------------------
\begin{equation}\lequ{S2BDF2fullL}\begin{split}
  \frac{1}{c}\frac{\Psi\iL^{\pm,k+1}-\Psi\iL^{\pm,n}}{\dt} = &
   -\sixth     \frac{2\mu^\pm}{h_i}\fn{\Psi^{\pm,n-1}_i - \Psi^{\pm,n-1}_{i-\half}}
   -\sixth     \frac{2\mu^\pm}{h_i}\fn{\Psi^{\pm,n}_i   - \Psi^{\pm,n}_{i-\half}}\\
  &-\frac{2}{3}\frac{2\mu^\pm}{h_i}\fn{\Psi^{\pm,k+1}_i - \Psi^{\pm,k+1}_{i-\half}}\\
  &-\sixth\sigma_{t,i,L}^{n-1}\Psi\iL^{\pm,n-1}
   -\sixth\sigma_{t,i,L}^{n}  \Psi\iL^{\pm,n}
   -\frac{2}{3}\sigma_{t,i,L}^k\Psi\iL^{\pm,k+1}\\
  &+\sixth\frac{\sigma_{s,i,L}^{n-1}}{2}\phi\iL^{n-1}
   +\sixth\frac{\sigma_{s,i,L}^{n}}{2}  \phi\iL^{n}
   +\frac{2}{3}\frac{\sigma_{s,i,L}^k}{2}\phi\iL^{k+1}\\
  &+\sixth\Q_{i,L}^{\pm,n-1}
   +\sixth\Q_{i,L}^{\pm,n}
   +\frac{2}{3}\Q_{i,L}^{\pm,k} \pec
\end{split}\end{equation}

\begin{equation}\lequ{S2BDF2fullR}\begin{split}
  \frac{1}{c}\frac{\Psi\iR^{\pm,k+1}-\Psi\iR^{\pm,n}}{\dt} = &
   -\sixth     \frac{2\mu^\pm}{h_i}\fn{\Psi^{\pm,n-1}_{i+\half} - \Psi^{\pm,n-1}_i}
   -\sixth     \frac{2\mu^\pm}{h_i}\fn{\Psi^{\pm,n}_{i+\half}   - \Psi^{\pm,n}_i}\\
  &-\frac{2}{3}\frac{2\mu^\pm}{h_i}\fn{\Psi^{\pm,k+1}_{i+\half} - \Psi^{\pm,k+1}_i}\\
  &-\sixth\sigma_{t,i,R}^{n-1}\Psi\iR^{\pm,n-1}
   -\sixth\sigma_{t,i,R}^{n}  \Psi\iR^{\pm,n}
   -\frac{2}{3}\sigma_{t,i,R}^k\Psi\iR^{\pm,k+1}\\
  &+\sixth\frac{\sigma_{s,i,R}^{n-1}}{2}\phi\iR^{n-1}
   +\sixth\frac{\sigma_{s,i,R}^{n}}{2}  \phi\iR^{n}
   +\frac{2}{3}\frac{\sigma_{s,i,R}^k}{2}\phi\iR^{k+1}\\
  &+\sixth\Q_{i,R}^{\pm,n-1}
   +\sixth\Q_{i,R}^{\pm,n}
   +\frac{2}{3}\Q_{i,R}^{\pm,k} \pec
\end{split}\end{equation}

% "if" block for developer mode
\iftoggle{DEVELOPERMODE}{

These equations can be solved by solving steady-state S$_2$ equations
of the form of Equations \requ{S2trL} and \requ{S2trR} by making
the following definitions:
\begin{equation}
  \tau = \frac{1}{c\dt} \pec \quad \alpha = \frac{2}{3} \pec
\end{equation}
\begin{multline}
  \tilde{\Q}_{i,L}^{\pm,k}
  = \frac{\Psi\iL^{\pm,n}}{c\dt}
  - \sixth\frac{2\mu^\pm}{h_i}\fn{\Psi_i^{\pm,n-1} - \Psi_{i-\half}^{\pm,n-1}}
  - \sixth\frac{2\mu^\pm}{h_i}\fn{\Psi_i^{\pm,n} - \Psi_{i-\half}^{\pm,n}}\\
  - \sixth\sigma_{t,i,L}^{n-1}\Psi\iL^{\pm,n-1}
  - \sixth\sigma_{t,i,L}^n    \Psi\iL^{\pm,n}
  + \sixth\frac{\sigma_{s,i,L}^{n-1}}{2}\phi\iL^{n-1}
  + \sixth\frac{\sigma_{s,i,L}^n}{2}    \phi\iL^n\\
  + \sixth\Q\iL^{\pm,n-1}
  + \sixth\Q\iL^{\pm,n}
  + \frac{2}{3}\Q\iL^{\pm,k}
\end{multline}
\begin{multline}
  \tilde{\Q}_{i,R}^{\pm,k}
  = \frac{\Psi\iR^{\pm,n}}{c\dt}
  - \sixth\frac{2\mu^\pm}{h_i}\fn{\Psi_{i+\half}^{\pm,n-1} - \Psi_i^{\pm,n-1}}
  - \sixth\frac{2\mu^\pm}{h_i}\fn{\Psi_{i+\half}^{\pm,n}   - \Psi_i^{\pm,n}}\\
  - \sixth\sigma_{t,i,R}^{n-1}\Psi\iR^{\pm,n-1}
  - \sixth\sigma_{t,i,R}^n    \Psi\iR^{\pm,n}
  + \sixth\frac{\sigma_{s,i,R}^{n-1}}{2}\phi\iR^{n-1}
  + \sixth\frac{\sigma_{s,i,R}^n}{2}    \phi\iR^n\\
  + \sixth\Q\iR^{\pm,n-1}
  + \sixth\Q\iR^{\pm,n}
  + \frac{2}{3}\Q\iR^{\pm,k}
\end{multline}

}{} % end of "if" block for developer mode; second pair of brackets is the "else" block

