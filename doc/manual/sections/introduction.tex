%===============================================================================
\section{Introduction}
%===============================================================================

In this work, a new IMEX scheme for solving the equations of radiation
hydrodynamics (RH) that is second-order accurate in both space and time is
presented and tested. A RH system combining a 1-D slab model
of compressible fluid dynamics with a grey radiation S$_2$ model is considered,
given by:
\begin{subequations}
\lequ{radhydro_system}
\be
\dxdy{\rho}{t}+\dxdy{}{x}\fn{\rho u} = 0 \pec
\lequ{cons_mass}
\ee 
\be
\dxdy{}{t}\fn{\rho u} + \dxdy{}{x}\fn{\rho u^2} + \dxdy{}{x}\fn{p}
  = \frac{\sigma_t}{c} \F_0 \pec
\lequ{cons_mom}
\ee
\be
\dxdy{E}{t} + \dxdy{}{x}\bracket{\fn{E+p}u}=-\sigma_a c \fn{aT^4 - \E}
  + \frac{\sigma_t u}{c} \F_0 \pec
\lequ{cons_energy}
\ee
\be
\frac{1}{c}\dxdy{\psi^+}{t} + \frac{1}{\sqrt{3}}\dxdy{\psi^+}{x}
  + \sigma_t \psi^+ = \frac{\sigma_s}{4\pi} c\E + \frac{\sigma_a}{4\pi} acT^4
  - \frac{\sigma_t u}{4\pi c} \F_0 + \frac{\sigma_t}{\sqrt{3}\pi}\E u
\pec
\lequ{intp}
\ee
\be
\frac{1}{c}\dxdy{\psi^-}{t} - \frac{1}{\sqrt{3}}\dxdy{\psi^-}{x}
  + \sigma_t \psi^- = \frac{\sigma_s}{4\pi} c\E + \frac{\sigma_a}{4\pi} acT^4
  - \frac{\sigma_t u}{4\pi c} \F_0 - \frac{\sigma_t}{\sqrt{3}\pi}\E u
\pec
\lequ{intm}
\ee
\end{subequations}
where $\rho$ is the density, $u$ is the velocity,
$E=\rho\fn{\frac{u^2}{2} + e}$ is the total material energy density,
$e$ is the specific internal energy density, $T$ is the material temperature,
$\E$ is the radiation energy density:
\be
\E = \frac{2\pi}{c}\fn{\psi^{+}+\psi^{-}} \pec
\lequ{Erad}
\ee
$\F$ is the radiation energy flux:
\be
\F = \frac{2\pi}{\sqrt{3}}\fn{\psi^{+}-\psi^{-}} \pec
\lequ{flux}
\ee
and $\F_0$ is an approximation to the comoving-frame flux,
\be
\lequ{F_nu_0}
\F_0 = \F-\frac{4}{3} \E u \pep
\ee
Note that multiplying Eqs.~\requ{intp} and \requ{intm} by $2\pi$ and summing
them gives the radiation energy equation:
\begin{subequations}
\be
\dxdy{\E}{t} + \dxdy{\F}{x} = \sigma_a c(aT^4 - \E) - \frac{\sigma_t u}{c}\F_0 \pec
\lequ{erad}
\ee
and multiplying \equ{intp} by $\frac{2\pi}{c\sqrt{3}}$ and \equ{intm} by
$-\frac{2\pi}{c\sqrt{3}}$ and sum them, we get the radiation momentum equation: 
\be
\frac{1}{c^2}\dxdy{\F}{t} + \frac{1}{3}\dxdy{\E}{x} = -\frac{\sigma_t}{c}\F_0 \pep
\ee
\end{subequations}
Equations \requ{cons_mass} through \requ{intm} are closed by assuming an ideal
equation of state (EOS):
\begin{subequations}
\be
p=\rho e (\gamma -1)
\lequ{pressure}
\pec
\ee
\be
T = \frac{e}{c_v} \pec
\lequ{matemp}
\ee
\end{subequations}
where $\gamma$ is the adiabatic index, and $c_v$ is the specific heat capacity.
However, the method presented is compatible with any valid EOS. 

%===============================================================================
\subsection{Source Definitions for the \texorpdfstring{S$_2$}{S-2} Equations}
%===============================================================================

Taking the zeroth angular moment of the S$_2$ equations, given by Equations
\requ{intp} and \requ{intm}, gives
\begin{equation}
\frac{1}{c}\dydt{\phi} + \dydx{\F} + \sigma_t\phi = \sigma_s\phi + Q_0 \pec
\lequ{zerothmoment}
\end{equation}
where $\phi = c\E$ and the source $Q_0$ is
\begin{equation}
Q_0 = \sigma_a acT^4 - \sigma_t\frac{u}{c}\F_0 \pep
\lequ{Q0}
\end{equation}
Taking the first angular moment of the S$_2$ equations gives
\begin{equation}
\frac{1}{c}\dydt{\F} + \third\dydx{\phi} + \sigma_t\F = Q_1 \pec
\lequ{firstmoment}
\end{equation}
where $Q_1$ is
\begin{equation}
Q_1 = \frac{4}{3}\sigma_t\E u \pep
\lequ{Q1}
\end{equation}
Defining a total source as
\begin{equation}
Q^\pm = Q_0 + 3\mu^\pm Q_1
\lequ{Qdef}
\end{equation}
and making the definitions
\begin{equation}
  \Psi^\pm = 2\pi\psi^\pm
\end{equation}
and
\begin{equation}
  \Q^\pm = \frac{Q^\pm}{2}
\end{equation}
and rewriting the S$_2$ equations gives
\begin{equation}\lequ{S2Q}
  \frac{1}{c}\dydt{\Psi^\pm} + \mu^\pm\dydx{\Psi^\pm} + \sigma_t\Psi^\pm
  = \frac{\sigma_s}{2}\phi + \Q^\pm \pec
\end{equation}
where $\mu^\pm=\pm\frac{1}{\sqrt{3}}$.

%===============================================================================
\subsection{Ensuring the Diffusion Limit}\label{diffusion_limit}
%===============================================================================

To ensure we get the diffusion limit, we need to have slopes for the internal energy
$e$ defined by the radiation
solver that are independent of the hydro solver.  In reality, in 1D this is not
necessary, but we will implement it as proof of principle for higher dimensions.   To mitigate confusion, in this
section, we will denote the standard hydro-state internal energy variables as $e$ and
the internal energy variables coming from the non-linear radiation solves as $e^r$. Care must be taken to
ensure that total energy is conserved, so we modify slopes in terms of the total
energy $E$, which inherently provides conservation by not modifying the cell averaged
total energy.  The modified slopes are only applied to the implicit terms. 

During the MUSCL solve, we
will just use the standard slope reconstruction formulas to advect variables to
state $U^*$ (or $U^{**}$).  Then, in the non-linear iteration loop for the radiation
and internal energy densities we use a modified slope for $E^*$, denoted $\Delta E^{r*}$, that will preserve the diffusion
limit.  This $\Delta E^{r*}$ is based on the edge values of $e^r$ of \emph{the last iteration of the previous
nonlinear radiation solve}.  For example, if we are solving the cycle 1 corrector from state $e^*$ to
$e^{n+1/2}$, we use the edge values of $e^r$ from the last iteration of the solve for
$e^{r,n+1/4}$ to construct the slopes for $\Delta E^{r,*}$. This does not change over
the duration of the radiation solve.  

In implementation, we store the necessary values of $e^r$ at edges to construct
$\Delta E^{*r}$ in the next solve (and thus the values $E^*_{R/L} = E^*_i \pm \frac{1}{2}\Delta E^{r*}$
needed for the LD radiation solve).  We approximate this slope based on the hydro values
for $E^*$ as:
\begin{equation}
    \Delta E^{r*} = E^{r*}_R - E^{r*}_L
\end{equation}
where 
\begin{equation}\label{estarr}
    E^{r*}_R = \rho^*_R\left((u_R^*)^2 + \left[e^*_i +
\frac{1}{2}(e_R^{r,n+1/4}-e_L^{r,n+1/4})\right]\right),
\end{equation}
\begin{equation}\label{estarl}
    E^{r*}_L = \rho^*_L\left((u_L^*)^2 + \left[e^*_i -
    \frac{1}{2}(e_R^{r,n+1/4}-e_L^{r,n+1/4})\right]\right),
\end{equation}
and subscript $i$ denotes cell average quantities.  The edge values of $\rho$ and
$u$ are formed as
\begin{equation}
    \rho^*_R = \rho^*_R + \frac{1}{2} \Delta \rho
\end{equation}
and
\begin{equation}
    u_R = \frac{(\rho u)^*_R + \frac{1}{2} \Delta(\rho u)}{\rho^*_R}
\end{equation}
where clearly the hydro slopes have came from the MUSCL solve. Once we have completed
the nonlinear solve for $e^{n+1/2}_{L/R}$ (here $e^{n+1/2}=e^{r,n+1/2}$), we must compute the change made to the
average total energy within the cell in such a way that total energy is conserved. The formula for the new total energy is
\begin{equation}\label{ei}
    E^{n+1/2}_i = \frac{1}{2}\left[\rho_L\left(\frac{1}{2}u_L^2 + e_L\right)
    +\rho_R\left(\frac{1}{2}u_R^2 + e_R\right)\right]^{n+1/2}
\end{equation}
where all variables are at time $t^{n+1/2}$. 

There is one issue with the approach outlined above.  Because we formulated
Eq.~\eqref{ei} to conserve total energy (as desired), $e_i\neq
\frac{1}{2}(e_L +e_R)$, in general.  This has the undesirable effect that if the radiation is uncoupled from
hydro ($\sigma_t =0$), you will not get the desired result of $\Delta E^{r*} = \Delta
E^{*}$. This can be seen from the bracketed terms in Eqs.~\eqref{estarl}
and~\eqref{estarr} and the fact that even though with the uncoupling
$e_{L/R}^{n+1/4,r} = e_{L/R}^{n+1/4}$, it is the case that $e_i^{r} =
\frac{1}{2}(e_L^r+e_R^r) \neq e_i$. In practice this was seen to have minimal effect
on results and I am not sure how to correct this.

TODO: An interesting test we never performed would be to use a `minmod' slope limiter
in a diffusive problem with the modified $e$-slopes and see if the radiation solution is
correct.  Just minmod limiting by itself will not preserve the diffusion limit and
the radiation solution will be artificially low.  The MMS diffusion limit test
demonstrates this behavior if you turn off the modified $e$ slopes and just use the
hydro provided $e^*$ slopes. This setting is at line 90 of the
\verb{nonlinearSolve.py{
module.


