%===============================================================================
\section{Introduction}
%===============================================================================

In this work, a new IMEX scheme for solving the equations of radiation
hydrodynamics (RH) that is second-order accurate in both space and time is
presented and tested. A RH system combining a 1-D slab model
of compressible fluid dynamics with a grey radiation S$_2$ model is considered,
given by:
\begin{subequations}
\lequ{radhydro_system}
\be
\dxdy{\rho}{t}+\dxdy{}{x}\fn{\rho u} = 0 \pec
\lequ{cons_mass}
\ee 
\be
\dxdy{}{t}\fn{\rho u} + \dxdy{}{x}\fn{\rho u^2} + \dxdy{}{x}\fn{p}
  = \frac{\sigma_t}{c} \F_0 \pec
\lequ{cons_mom}
\ee
\be
\dxdy{E}{t} + \dxdy{}{x}\bracket{\fn{E+p}u}=-\sigma_a c \fn{aT^4 - \E}
  + \frac{\sigma_t u}{c} \F_0 \pec
\lequ{cons_energy}
\ee
\be
\frac{1}{c}\dxdy{\psi^+}{t} + \frac{1}{\sqrt{3}}\dxdy{\psi^+}{x}
  + \sigma_t \psi^+ = \frac{\sigma_s}{4\pi} c\E + \frac{\sigma_a}{4\pi} acT^4
  - \frac{\sigma_t u}{4\pi c} \F_0 + \frac{\sigma_t}{\sqrt{3}\pi}\E u
\pec
\lequ{intp}
\ee
\be
\frac{1}{c}\dxdy{\psi^-}{t} - \frac{1}{\sqrt{3}}\dxdy{\psi^-}{x}
  + \sigma_t \psi^- = \frac{\sigma_s}{4\pi} c\E + \frac{\sigma_a}{4\pi} acT^4
  - \frac{\sigma_t u}{4\pi c} \F_0 - \frac{\sigma_t}{\sqrt{3}\pi}\E u
\pec
\lequ{intm}
\ee
\end{subequations}
where $\rho$ is the density, $u$ is the velocity,
$E=\rho\fn{\frac{u^2}{2} + e}$ is the total material energy density,
$e$ is the specific internal energy density, $T$ is the material temperature,
$\E$ is the radiation energy density:
\be
\E = \frac{2\pi}{c}\fn{\psi^{+}+\psi^{-}} \pec
\lequ{Erad}
\ee
$\F$ is the radiation energy flux:
\be
\F = \frac{2\pi}{\sqrt{3}}\fn{\psi^{+}-\psi^{-}} \pec
\lequ{flux}
\ee
and $\F_0$ is an approximation to the comoving-frame flux,
\be
\lequ{F_nu_0}
\F_0 = \F-\frac{4}{3} \E u \pep
\ee
Note that multiplying Eqs.~\requ{intp} and \requ{intm} by $2\pi$ and summing
them gives the radiation energy equation:
\begin{subequations}
\be
\dxdy{\E}{t} + \dxdy{\F}{x} = \sigma_a c(aT^4 - \E) - \frac{\sigma_t u}{c}\F_0 \pec
\lequ{erad}
\ee
and multiplying \equ{intp} by $\frac{2\pi}{c\sqrt{3}}$ and \equ{intm} by
$-\frac{2\pi}{c\sqrt{3}}$ and sum them, we get the radiation momentum equation: 
\be
\frac{1}{c^2}\dxdy{\F}{t} + \frac{1}{3}\dxdy{\E}{x} = -\frac{\sigma_t}{c}\F_0 \pep
\ee
\end{subequations}
Equations \requ{cons_mass} through \requ{intm} are closed by assuming an ideal
equation of state (EOS):
\begin{subequations}
\be
p=\rho e (\gamma -1)
\lequ{pressure}
\pec
\ee
\be
T = \frac{e}{c_v} \pec
\lequ{matemp}
\ee
\end{subequations}
where $\gamma$ is the adiabatic index, and $c_v$ is the specific heat capacity.
However, the method presented is compatible with any valid EOS. 

%===============================================================================
\subsection{Source Definitions for the \texorpdfstring{S$_2$}{S-2} Equations}
%===============================================================================

Taking the zeroth angular moment of the S$_2$ equations, given by Equations
\requ{intp} and \requ{intm}, gives
\begin{equation}
\frac{1}{c}\dydt{\phi} + \dydx{\F} + \sigma_t\phi = \sigma_s\phi + Q_0 \pec
\lequ{zerothmoment}
\end{equation}
where $\phi = c\E$ and the source $Q_0$ is
\begin{equation}
Q_0 = \sigma_a acT^4 - \sigma_t\frac{u}{c}\F_0 \pep
\lequ{Q0}
\end{equation}
Taking the first angular moment of the S$_2$ equations gives
\begin{equation}
\frac{1}{c}\dydt{\F} + \third\dydx{\phi} + \sigma_t\F = Q_1 \pec
\lequ{firstmoment}
\end{equation}
where $Q_1$ is
\begin{equation}
Q_1 = \frac{4}{3}\sigma_t\E u \pep
\lequ{Q1}
\end{equation}
Defining a total source as
\begin{equation}
Q^\pm = Q_0 + 3\mu^\pm Q_1
\lequ{Qdef}
\end{equation}
and making the definitions
\begin{equation}
  \Psi^\pm = 2\pi\psi^\pm
\end{equation}
and
\begin{equation}
  \Q^\pm = \frac{Q^\pm}{2}
\end{equation}
and rewriting the S$_2$ equations gives
\begin{equation}\lequ{S2Q}
  \frac{1}{c}\dydt{\Psi^\pm} + \mu^\pm\dydx{\Psi^\pm} + \sigma_t\Psi^\pm
  = \frac{\sigma_s}{2}\phi + \Q^\pm \pec
\end{equation}
where $\mu^\pm=\pm\frac{1}{\sqrt{3}}$.

%===============================================================================
\subsection{Ensuring the Diffusion Limit}\label{diffusion_limit}
%===============================================================================

To ensure we get the diffusion limit, we need to have slopes defined by the radiation
solver that are independent of the hydro solver.  In reality, in 1D this is not
necessary, but we will implement it as proof of principle for higher dimensions. Because we need to use
old time step values in building our systems, it will be necessary to store the
``radiation" slopes from the previous solves. For example, during the MUSCL solve, we
will just use the standard MUSCL formula and no modifications to the slope. Thus,
E$^*$ is evaluated when evaluating the new
total energies in the equations, we will use the following formula,

