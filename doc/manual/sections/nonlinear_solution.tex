%===============================================================================
\section{Solution of Nonlinear Equations}
%===============================================================================
\subsection{Linearization of Material Energy Equation}\label{sec:linearization}
%===============================================================================

Within each solution time step, first the hydro variables are advected (either
using local predicted fluxes or a Riemann solver).  Then, a non-linear system
must be solved.  Consider the case of the non-linear system to be solved for
Crank Nicolson over a time step from $t_n$ to $t_{n+1}$.  The changes to the
non-linear system for the predictor and corrector time steps will only affect
the choice of $\dt$, the scale factor ($1/2$ for CN) on implicit terms, the end
time state, and the known source term $Q_E^k$ yet to be defined on the right
hand side. All terms in $Q_k$ are known from previous states in time or
iteration $k$.  The non-linear equations to be solved in this case are,
dropping spatial differencing indices,

\momentumUpdateCN{n}{}{}{\lequ{vel_update}}
\energyUpdateCN{n}{}{}{\lequ{mat}}

plus the $S_2$ equations.  To simplify the algebra, a source term $Q_E$ is
defined for all the known, lagged quantities in the above equation as
\begin{multline}
   Q_E^{k} = -\half\left[\sa c\left( aT^4 - \E\right)\right]^n\\
   +\CN{\sigma_t \frac{u}{c} \left( \F - \frac{4}{3} \E u \right)}{n}{k} \pep
\end{multline}
First, Eq.~\eqref{eq:vel_update} is solved for a new velocity, using lagged
radiation energy and flux densities.  For the initial solve, these can be taken
at $t_n$.  We then linearize the Planckian function about some temperature near
$T^{k+1}$, denoted $T^k$. The linearized Planckian is
\begin{equation}
  \sigma_a^k a c \left( T^{k+1} \right)^4 = 
  \sigma_a^k a c \left[(T^k)^4 + \frac{4(T^k)^3}{c_v^k}\left(
  e^{k+1} - e^{k}  \right)\right] \pep
    \label{eq:linearize}
\end{equation}
For the initial iteration, $T^k=T^n$.  The above equation is substituted into
Eq.~\eqref{eq:mat} and we define $\beta^k=\frac{4a(T^k)^3}{c_v^k}$ for clarity.
The resulting equation can be solved for $(e^{k+1} - e^{k})$ through
algebraic manipulation:
\begin{align*}
   \frac{E^{k+1} - E^*}{\dt} &= - \frac{1}{2}\left[\sa^k c \left(
   a(T^{k+1})^4 - \E^{k+1}\right)\right]+Q_E^k \\
   \frac{E^{k+1} - E^*}{\dt} &= - \frac{1}{2}\left[\sa^k c \left(
   a(T^k)^4 + \beta^k(e^{k+1} - e^k)  - \E^{k+1}\right)\right]+Q_E^k \pec
\end{align*}
Adding and subtracting $\frac{\rho^{n+1}e^k}{\dt}$ on the left hand side gives
\begin{multline*}
   \frac{E^{k+1}-\rho^{n+1} e^k + \rho^{n+1} e^k - E^*}{\dt} = \\
   -\frac{1}{2}\left[\sa^{k}c\left(
   a(T^k)^4 + \beta^k(e^{k+1} - e^k)  - \E^{k+1}\right)\right]+Q_E^k \pep
\end{multline*}
The superscript on $\rho$ is dropped because $\rho^{n+1} = \rho^*$. Then, using
$E=\rho\fn{\frac{u^2}{2} + e}$, the left hand side can be simplified as 
\begin{multline} 
   \frac{E^{k+1} - \rho e^k + \rho e^k - E^*}{\dt} =\\
   \frac{\rho}{\dt}\left[(e^{k+1} - e^{k}) + \frac{1}{2}((u^{k+1})^2-(u^*)^2) +
   (e^k-e^*) \right] \pep
\end{multline}
Solving for $(e^{k+1} - e^{k})$ then gives
\begin{multline}
    e^{k+1}-e^k=\\
    \frac{{\dt}\left[ \frac{1}{2}\sa^k c (\E^{k+1} -
    a(T^k)^4)+Q_E^k \right]- \rho(e^k-e^*)-
    \frac{\rho}{2}((u^{k+1})^2-(u^*)^2)}
    {\rho +\frac{1}{2}\sa^k c \dt\beta^k} \pep
\label{eq:energy_update}
\end{multline}
We then multiply the above equation by $\sigma_a^k c \beta^k$ and divide the RHS
by $\rho/\rho$; this will simplify substitution back into
Eq.~\eqref{eq:linearize}.
\begin{multline}
    \sa^k c \beta^k (e^{k+1} - e^k) = \frac{\frac{1}{2}\sigma_a^k c \dt
    \frac{\beta^k}{\rho}}{1+\frac{1}{2}\sa^k c \dt \frac{\beta^k}{\rho}}
    \left(\sa^k c\left[\E^{k+1} - a (T^k)^4\right] + 2Q_E^k\right)\\
    - \left(\frac{\frac{1}{2}\sa^k c\dt \frac{\beta^k}{\rho}}
    {1 +\frac{1}{2}\sa^k c \dt \frac{\beta^k}{\rho}}\right)
    \left(\frac{2\rho}{\dt}\right)\left[(e^k - e^*)
    + \frac{1}{2}((u^{k+1})^2 - (u^*)^2)\right] \pep
 \end{multline}
The effective scattering fraction $\nu_{1/2}$, for the case of Crank Nicolson, is
defined as
\begin{equation}
    \nu_{1/2}^k = \frac{\half\sigma_a^k c\dt \frac{\beta^k}{\rho}}
    {1 + \half\sa^k c\dt \frac{\beta^k}{\rho}  } \pep
\end{equation}
Making this substitution gives
\begin{multline*}
    \sa^k c \beta^k(e^{k+1} - e^k) = \nu_{1/2}^k \left(\sa^k c
    \left[\E^{k+1} - a (T^k)^4\right] + 2Q_E^k\right)\\
    -\frac{2\nu_{1/2}^k\rho}{\dt} \left[(e^k - e^*)
    + \frac{1}{2}((u^{k+1})^2 - (u^*)^2)\right] \pep
\end{multline*}
Finally, this can be substituted into Eq.~\eqref{eq:linearize} and the $(T^k)^4$
terms simplified, giving the source term
\begin{multline}
   \sigma_a^k a c\left(T^{k+1}\right)^4 = \left(1 - \nu_{1/2}^k\right)
   \sigma_a^k a c (T^k)^4 + \sigma_a^k c \nu_{1/2}^k \E^{k+1}\\
   +2\nu_{1/2}^k Q_E^k - \frac{2\rho\nu_{1/2}^k}{\dt}
   \left[(e^k - e^*) + \frac{1}{2}((u^{k+1})^2 - (u^*)^2)\right] \pep
\end{multline}
The above expression can be substituted for the emission source in the S$_2$
equations, including an effective scattering cross section given by $\sigma_a^k
\nu_{1/2}^k$. It is important to note there is no CN scale factor of 1/2 out front of
this term, so that factor is still included in the substitution into the S$_2$
equations.

After solving for $\E^{k+1}$, a new internal energy can be estimated
using Eq.~\eqref{eq:energy_update}. It is important to use this linearized
equation to ensure energy conservation.  We manipulate Eq.~\eqref{eq:energy_update}
to simplify the expression.  We use the fact
that
\begin{equation}
   (1 - \nu_{1/2}) = \frac{1}{1+\frac{1}{2}\sigma_a^kc\Delta_t\frac{\beta^k}{\rho}}.
\end{equation}
which simplifies the equation for the new internal energy $e^{k+1}$ to
\begin{multline}
    e^{k+1} = \frac{1}{2}\frac{(1-\nu_{1/2})}{\rho}\left[\sigma_a^k c\Delta t \left(c\E
    - ac(T^k)^4\right) + 2\Delta t Q_E^k \right] \\ - (1 -
    \nu_{1/2})\left(\frac{1}{2}[(u^{k+1})^2 - (u^{k})^2]-e^*\right) + \nu_{1/2}e^k
\end{multline}

This process can now be repeated until
convergence, beginning with a solve of Eq.~\eqref{eq:vel_update} with updated
radiation quantities.  Once the system is converged, the EOS can be used to
update $p^{n+1}$.

%===============================================================================
\subsection{Generalization to Other Temporal Discretizations}
%===============================================================================

Section \ref{sec:linearization} derived the linearization solution process for
the Crank-Nicolson (CN) scheme. In this section, this process is generalized to
other temporal discretizations, namely Backward Euler (BE) and BDF2
discretizations.  First, the original equation is rewritten as
\begin{align*}
   \frac{E^{k+1} - E^*}{\dt} &= - \alpha \left[\sa^k c \left(
   a(T^{k+1})^4 - \E^{k+1}\right)\right]+Q_E^k \\
\end{align*}
where for BDF2,
\begin{multline}
    Q^{k}_E = -\frac{1}{6}\left[\sa c\left( aT^4 - \E\right)\right]^{n-1}
    -\frac{1}{6}\left[\sa c\left( aT^4 - \E\right)\right]^{n} \\
    +\BDF{\sigma_t \frac{u}{c} \left( \F - \frac{4}{3} \E u \right)}{n-1}{n}{k},
\end{multline}
for CN,
\begin{multline}
    Q^{k}_E = -\half\left[\sa c\left( aT^4 - \E\right)\right]^n\\
   +\CN{\sigma_t \frac{u}{c} \left( \F - \frac{4}{3} \E u \right)}{n}{k},
\end{multline}
and for BE,
\begin{equation*}
    Q^{k}_E = \sigma_t \frac{u}{c} \left( \F - \frac{4}{3} \E u \right)^k.
\end{equation*}
The scale factor $\alpha$ for BE, CN, and BDF2 is 1, $\half$, and $\frac{2}{3}$,
respectively.  With these definitions, the Planckian source term becomes
\begin{multline}
   \sigma_a^k a c\left(T^{k+1}\right)^4 = \left(1 - \nu_{\alpha}^k\right)
   \sigma_a^k a c (T^k)^4 + \sigma_a^k c \nu_{\alpha}^k \E^{k+1}\\
   +\frac{\nu_{\alpha}^k}{\alpha} Q_E^k - \frac{\rho\nu_{\alpha}^k}{\alpha\dt}
   \left[(e^k - e^*) + \frac{1}{2}((u^{k+1})^2 - (u^*)^2)\right] \pep
\end{multline}
with 
\begin{equation}
    \nu^k_{\alpha} = \frac{\alpha\sigma_a c\Delta t \frac{\beta^k}{\rho}}{1 +
    \alpha\sigma_a c\Delta t \frac{\beta^k}{\rho}}.
\end{equation}
The energy update equation becomes
\begin{multline}
    e^{k+1} = \alpha\frac{(1-\nu_{\alpha})\Delta t}{\rho}\left[\sigma_a^k c \left(
    \E^{k+1} - a(T^k)^4\right) + \frac{Q_E^k}{\alpha} \right] \\ - (1 -
    \nu_{\alpha})\left(\frac{1}{2}[(u^{k+1})^2 - (u^*)^2]\right)
    +(1-\nu_\alpha)e^*+ \nu_{\alpha}e^k \pep
\end{multline}

