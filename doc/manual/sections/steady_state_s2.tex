%===============================================================================
\subsubsection{Steady-State}\lsec{S2ssdisc}
%===============================================================================
Using an LLD discretization, the steady-state S$_2$ equations, obtained by
dropping the $\dydt{\Psi^\pm}$ term in Equation \requ{S2Q}, become
\begin{equation}\lequ{S2ssL}
  \frac{2\mu^\pm}{h_i}\fn{\Psi^{\pm,k+1}_i - \Psi^{\pm,k+1}_{i-\half}}
  +\sigma_{t,i,L}^k\Psi\iL^{\pm,k+1}
  -\frac{\sigma_{s,i,L}^k}{2}\phi\iL^{k+1}
  =\Q_{i,L}^{\pm,k} \pec
\end{equation}
\begin{equation}\lequ{S2ssR}
  \frac{2\mu^\pm}{h_i}\fn{\Psi^{\pm,k+1}_{i+\half} - \Psi^{\pm,k+1}_i}
  +\sigma_{t,i,R}^k\Psi\iR^{\pm,k+1}
  -\frac{\sigma_{s,i,R}^k}{2}\phi\iR^{k+1}
  =\Q_{i,R}^{\pm,k} \pep
\end{equation}

Coefficients of the resulting linear system matrix are the following,
where conditions in parentheses indicate when a term does not
apply for the boundary elements $i=1$ and $i=N$:
\begin{center}
\begin{tabular}{|l||c|c|c|}\hline
          & $i-1,R,+$ & $i,L,-$ & $i,L,+$\\\hline\hline
  $i,L,-$ &
          & $-\frac{\mu^-}{h_i} + \sigtL - \sigsL$
          & $-\sigsL$ \\\hline
  $i,L,+$ & $-\frac{2\mu^+}{h_i}$ ($i\ne 1$)
          & $-\sigsL$
          & $\frac{\mu^+}{h_i} + \sigtL - \sigsL$ \\\hline
  $i,R,-$ &
          & $-\frac{\mu^-}{h_i}$
          & \\\hline
  $i,R,+$ &
          &
          & $-\frac{\mu^+}{h_i}$ \\\hline
\end{tabular}
\end{center}
\begin{center}
\begin{tabular}{|l||c|c|c|}\hline
          & $i,R,-$ & $i,R,+$ & $i+1,L,-$\\\hline\hline
  $i,L,-$ & $\frac{\mu^-}{h_i}$
          &
          & \\\hline
  $i,L,+$ &
          & $\frac{\mu^+}{h_i}$
          & \\\hline
  $i,R,-$ & $-\frac{\mu^-}{h_i} + \sigtR - \sigsR$
          & $-\sigsR$
          & $\frac{2\mu^-}{h_i}$ ($i\ne N$) \\\hline
  $i,R,+$ & $-\sigsR$
          & $\frac{\mu^+}{h_i} + \sigtR - \sigsR$
          & \\\hline
\end{tabular}
\end{center}
The right hand side vector for the linear system is the following,
where the columns indicate to which cell(s) the expressions
correspond:

\begin{center}
\begin{tabular}{|l||c|c|c|}\hline
          & $i=1$ & $i=2\ldots N-1$ & $i=N$\\\hline\hline
  $i,L,-$ & $\Q^-_{i,L}$
          & $\Q^-_{i,L}$
          & $\Q^-_{i,L}$ \\\hline
  $i,L,+$ & $\Q^+_{i,L} + \frac{2\mu^+}{h_i}\Psi^+_{inc}$
          & $\Q^+_{i,L}$
          & $\Q^+_{i,L}$ \\\hline
  $i,R,-$ & $\Q^-_{i,R}$
          & $\Q^-_{i,R}$
          & $\Q^-_{i,R} - \frac{2\mu^-}{h_i}\Psi^-_{inc}$ \\\hline
  $i,R,+$ & $\Q^+_{i,R}$
          & $\Q^+_{i,R}$
          & $\Q^+_{i,R}$ \\\hline
\end{tabular}
\end{center}

The incoming boundary fluxes $\Psi^\pm_{inc}$ are computed from the
user-supplied boundary half-range currents $j^\pm$ using quadrature:
\begin{equation}
   j^+(x_L) = 2\pi\int\limits_0^1 \psi(x_L,\mu)\mu d\mu
   \approx \sum\limits_{\mu_d>0}\psi(x_L,\mu_d)\mu_d w_d
   = \mu^+\Psi^+_{inc} \pec
\end{equation}
\begin{equation}
   j^-(x_R) = -2\pi\int\limits_{-1}^0 \psi(x_R,\mu)\mu d\mu
   \approx -\sum\limits_{\mu_d<0}\psi(x_R,\mu_d)\mu_d w_d
   = -\mu^-\Psi^-_{inc} \pep
\end{equation}

Transient S$_2$ solvers can employ a steady-state solver if
the following generalization is made to Equations \requ{S2ssL}
and \requ{S2ssR}:
\begin{equation}\lequ{S2trL}
  \alpha\frac{2\mu^\pm}{h_i}\fn{\Psi^{\pm,k+1}_i - \Psi^{\pm,k+1}_{i-\half}}
  +\fn{\alpha\sigma_{t,i,L}^k + \tau}\Psi\iL^{\pm,k+1}
  -\alpha\frac{\sigma_{s,i,L}^k}{2}\phi\iL^{k+1}
  =\tilde{\Q}_{i,L}^{\pm,k} \pec
\end{equation}
\begin{equation}\lequ{S2trR}
  \alpha\frac{2\mu^\pm}{h_i}\fn{\Psi^{\pm,k+1}_{i+\half} - \Psi^{\pm,k+1}_i}
  +\fn{\alpha\sigma_{t,i,R}^k + \tau}\Psi\iR^{\pm,k+1}
  -\alpha\frac{\sigma_{s,i,R}^k}{2}\phi\iR^{k+1}
  =\tilde{\Q}_{i,R}^{\pm,k} \pec
\end{equation}
where $\alpha$, $\tau$, and $\tilde{\Q}^\pm$ depend on the time
discretization used.

