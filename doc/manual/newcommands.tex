% Mark URL's
\newcommand{\URL}[1]{{\textcolor{blue}{#1}}}
%
% Ways of grouping things
%
\newcommand{\bracket}[1]{\left[ #1 \right]}
\newcommand{\bracet}[1]{\left\{ #1 \right\}}
\newcommand{\fn}[1]{\left( #1 \right)}
\newcommand{\ave}[1]{\left\langle #1 \right\rangle}
\newcommand{\norm}[1]{\Arrowvert #1 \Arrowvert}
\newcommand{\abs}[1]{\arrowvert #1 \arrowvert}
%
% Derivative forms
%
\newcommand{\dxdy}[2]{\frac{\partial #1}{\partial #2}}
\newcommand{\dxy}[2]{\frac{d #1}{d #2}}
\newcommand{\dydx}[1]{\frac{\partial #1}{\partial x}}
\newcommand{\dydt}[1]{\frac{\partial #1}{\partial t}}
\newcommand{\dxdz}[1]{\frac{\partial #1}{\partial z}}
\newcommand{\dfdt}[1]{\frac{\partial}{\partial t} \fn{#1}}
\newcommand{\dfdz}[1]{\frac{\partial}{\partial z} \fn{#1}}
\newcommand{\ddt}[1]{\frac{\partial}{\partial t} #1}
\newcommand{\ddz}[1]{\frac{\partial}{\partial z} #1}
\newcommand{\dd}[2]{\frac{\partial}{\partial #1} #2}
\newcommand{\ddx}[1]{\frac{\partial}{\partial x} #1}
\newcommand{\ddy}[1]{\frac{\partial}{\partial y} #1}
\newcommand{\dxdyn}[3]{\frac{\partial ^{#3} #1 }{\partial #2 ^{#3}}}
\newcommand{\Dxdy}[2]{\frac{D #1}{D #2}}
\newcommand{\Dxy}[2]{\frac{D #1}{D #2}}
%
% Vector forms
%
\renewcommand{\vec}[1]{\mbox{$\stackrel{\longrightarrow}{#1}$}}
\renewcommand{\div}{\mbox{$\vec{\nabla} \cdot$}}
\newcommand{\grad}{\mbox{$\vec{\nabla}$}}
\newcommand{\bb}[1]{\bar{\bar{#1}}}
%
% Equation beginnings and endings
%
\newcommand{\bea}{\begin{eqnarray}}
\newcommand{\eea}{\end{eqnarray}}
\newcommand{\be}{\begin{equation}}
\newcommand{\ee}{\end{equation}}
\newcommand{\beas}{\begin{eqnarray*}}
\newcommand{\eeas}{\end{eqnarray*}}
\newcommand{\bdm}{\begin{displaymath}}
\newcommand{\edm}{\end{displaymath}}
%
% Equation punctuation
%
\newcommand{\pec}{\, ,}
\newcommand{\pep}{\, .} 
\newcommand{\pev}{\hspace{0.25in}}
%
% Equation labels and references, figure references, table references
%
\newcommand{\lequ}[1]{\label{eq:#1}}
\newcommand{\equ}[1]{Eq.~(\ref{eq:#1})}
\newcommand{\equs}[1]{Eqs.~(\ref{eq:#1})}
\newcommand{\requ}[1]{(\ref{eq:#1})}
\newcommand{\lfig}[1]{\label{fi:#1}}
\newcommand{\fig}[1]{Fig.~\ref{fi:#1}}
\newcommand{\figs}[1]{Figs.~\ref{fi:#1}}
\newcommand{\rfig}[1]{\ref{fi:#1}}
\newcommand{\lta}[1]{\label{ta:#1}}
\newcommand{\ta}[1]{Table~\ref{ta:#1}}
\newcommand{\rta}[1]{\ref{ta:#1}}
\newcommand{\lsec}[1]{\label{sec:#1}}
\newcommand{\rsec}[1]{\ref{sec:#1}}
%
% Superscript and subscript in text
%
\newcommand{\supertext}[1]{\ensuremath{^{\textrm{#1}}}}
\newcommand{\subtext}[1]{\ensuremath{_{\textrm{#1}}}}
%
% List beginnings and endings
%
\newcommand{\bl}{\bss\begin{itemize}}
\newcommand{\el}{\vspace{-.5\baselineskip}\end{itemize}\ess}
\newcommand{\ben}{\bss\begin{enumerate}}
\newcommand{\een}{\vspace{-.5\baselineskip}\end{enumerate}\ess}
%
% Figure and table beginnings and endings
%
\newcommand{\bfg}{\begin{figure}}
\newcommand{\efg}{\end{figure}}
\newcommand{\bt}{\begin{table}}
\newcommand{\et}{\end{table}}
%
% Tabular and center beginnings and endings
%
\newcommand{\bc}{\begin{center}}
\newcommand{\ec}{\end{center}}
\newcommand{\btb}{\begin{center}\begin{tabular}}
\newcommand{\etb}{\end{tabular}\end{center}}
%
% Single space command
%
\newcommand{\bss}{\begin{singlespace}}
\newcommand{\ess}{\end{singlespace}}
%
% Quick commands for symbols
%
\newcommand{\half}{\frac{1}{2}}
\newcommand{\third}{\frac{1}{3}}
\newcommand{\twothird}{\frac{2}{3}}
\newcommand{\fourth}{\frac{1}{4}}
\newcommand{\sixth}{\frac{1}{6}}
\newcommand{\mdot}{\dot{m}}
%\newcommand{\ten}[1]{\times 10^{#1}\,}
\newcommand{\cL}{{\cal L}}
\newcommand{\cD}{{\cal D}}
\newcommand{\cF}{{\cal F}}
\newcommand{\cE}{{\cal E}}
\renewcommand{\Re}{\mbox{Re}}
\newcommand{\Ma}{\mbox{Ma}}
\newcommand{\mA}{\mathbf{A}}
\newcommand{\mB}{\mathbf{B}}
\newcommand{\mC}{\mathbf{C}}
\newcommand{\E}{\mathcal{E}}
\newcommand{\F}{\mathcal{F}}
\newcommand{\Q}{\mathcal{Q}}
\newcommand{\U}{\mathbf{U}}
\renewcommand{\H}{\mathbf{H}}
\newcommand{\R}{\mathbf{R}}
\newcommand{\Flux}{\mathbf{F}}
\newcommand{\dt}{\Delta t}
\newcommand{\dx}{\Delta x}
\newcommand{\iL}{_{i,L}}
\newcommand{\iR}{_{i,R}}
\newcommand{\sa}{\sigma_a}
\newcommand{\sigsL}{\frac{\sigma_{s,i,L}^k}{2}}
\newcommand{\sigsR}{\frac{\sigma_{s,i,R}^k}{2}}
\newcommand{\sigtL}{\sigma_{t,i,L}^k}
\newcommand{\sigtR}{\sigma_{t,i,R}^k}
\newcommand{\halfh}{\frac{h_i}{2}}
\newcommand{\CN}[3]{\half\left[#1\right]^#2 + \half\left[#1\right]^#3}
\newcommand{\CNN}[3]{\half\left[#1\right]^#2 - \half\left[#1\right]^#3}
\newcommand{\BDF}[4]{\sixth\left[#1\right]^{#2} + \sixth\left[#1\right]^{#3} + \twothird\left[#1\right]^{#4}}
%
% More Quick Commands
%
\newcommand{\bi}{\begin{itemize}}
\newcommand{\ei}{\end{itemize}}
\newcommand{\dxi}{\Delta x_i}
\newcommand{\dyj}{\Delta y_j}
\newcommand{\ts}[1]{\textstyle #1}
%
% Equations
%
\newcommand{\momentumSource}{
   \left[\frac{\sigma_{t}}{c}\left(\F-\frac{4}{3}\E u\right)\right]
}

% #1: old   time index
% #2: coefficient for time step size
% #3: spatial differencing subscript command
% #4: optional label command
\newcommand{\momentumUpdateCN}[4]{
\begin{equation}
  \frac{\rho^*#3\left(u^{k+1}#3-u^*#3\right)}{#2\dt} = 
   \half\momentumSource^{#1}#3
  +\half\momentumSource^k#3
  \pep
#4
\end{equation}
}

% #1: older time index
% #2: old   time index
% #3: coefficient for time step size
% #4: spatial differencing subscript command
% #5: optional label command
\newcommand{\momentumUpdateBDFTwo}[5]{
\begin{equation}\begin{split}
  \frac{\rho^*#4\left(u^{k+1}#4-u^*#4\right)}{#3\dt} =  
  & \sixth\momentumSource^{#1}#4
   +\sixth\momentumSource^{#2}#4\\
  &+\frac{2}{3}\momentumSource^k#4
  \pep
#5
\end{split}\end{equation}
}

\newcommand{\energyEmissionSource}{
   \left[\sigma_a c\left(aT^4 - \E\right)\right]
}

\newcommand{\energyEmissionSourceNew}{
   \sigma_a^k c\left[aT^4 - \E\right]^{k+1}
}

\newcommand{\energyDriftSource}{
   \left[\sigma_t\frac{u}{c}\left(\F-\frac{4}{3}\E u\right)\right]
}

% #1: old   time index
% #2: coefficient for time step size
% #3: spatial differencing subscript command
% #4: optional label command
\newcommand{\energyUpdateCN}[4]{
\begin{equation}\begin{split}
  \frac{E^{k+1}#3-E^*#3}{#2\dt} = &
  -\half\energyEmissionSource^{#1}#3
  -\half\energyEmissionSourceNew#3\\
  &+\half\energyDriftSource^{#1}#3
   +\half\energyDriftSource^k#3
  \pep
#4
\end{split}\end{equation}
}

% #1: older time index
% #2: old   time index
% #3: coefficient for time step size
% #4: spatial differencing subscript command
% #5: optional label command
\newcommand{\energyUpdateBDFTwo}[5]{
\begin{equation}\begin{split}
  \frac{E^{k+1}#4-E^*#4}{#3\dt} = &
  -\sixth\energyEmissionSource^{#1}#4
  -\sixth\energyEmissionSource^{#2}#4\\
  &-\frac{2}{3}\energyEmissionSourceNew#4
   +\sixth\energyDriftSource^{#1}#4\\
  &+\sixth\energyDriftSource^{#2}#4
   +\frac{2}{3}\energyDriftSource^k#4
  \pep
#5
\end{split}\end{equation}
}

% #1: old time index
% #2: new time index
% #3: coefficient for time step size
% #4: optional label command
\newcommand{\hydroPredictor}[4]{
\begin{equation}#4
  \H_i^{#2} = \H_i^{#1} - \frac{#3\dt}{\dx}
  \left(\Flux(\H\iR^{#1}) - \Flux(\H\iL^{#1})\right) \pep
\end{equation}
}

% #1: older time index
% #2: old time index
% #3: new time index
% #4: coefficient for time step size
% #5: optional label command
\newcommand{\hydroCorrector}[5]{
\begin{equation}#5
  \H_i^{#3} = \H_i^{#1} - \frac{#4\dt}{\dx}
  \left(\Flux_{i+\half}^{#2} - \Flux_{i-\half}^{#2}\right) \pep
\end{equation}
}

% #1: time index
\newcommand{\slopeEquations}[1]{
\begin{equation}
  \Delta_i^{#1} = \half\Delta\H_{i-\half}^{#1} + \half\Delta\H_{i+\half}^{#1} \pec
\end{equation}
\begin{equation}
  \Delta\H_{i-\half}^{#1} = \H_i^{#1} - \H_{i-1}^{#1} \pec\quad
  \Delta\H_{i+\half}^{#1} = \H_{i+1}^{#1} - \H_i^{#1} \pep
\end{equation}
}

% #1: time index
\newcommand{\hydroLinearRepresentation}[1]{
\begin{equation}
  \H\iL^{#1} = \H_i^{#1} - \frac{\Delta_i^{#1}}{2} \pec
  \quad
  \H\iR^{#1} = \H_i^{#1} + \frac{\Delta_i^{#1}}{2} \pep
\end{equation}
}
