\documentclass[preprint,12pt]{elsarticle}
\usepackage{amssymb}
\usepackage{amsmath}
\usepackage{color}
% Mark URL's
\newcommand{\URL}[1]{{\textcolor{blue}{#1}}}

%\newcommand{\elfit}{\ensuremath{\operatorname{Im}(-1/\epsilon(\vq,\omega)}}
%\msection{}-->section commands
%\tradem{}  -->add TM subscript to entry
%\ucatm{}   -->add trademark footnote about entry
%
% Ways of grouping things
%
\newcommand{\bracket}[1]{\left[ #1 \right]}
\newcommand{\bracet}[1]{\left\{ #1 \right\}}
\newcommand{\fn}[1]{\left( #1 \right)}
\newcommand{\ave}[1]{\left\langle #1 \right\rangle}
\newcommand{\norm}[1]{\Arrowvert #1 \Arrowvert}
\newcommand{\abs}[1]{\arrowvert #1 \arrowvert}

%
% Derivative forms
%
\newcommand{\dx}[1]{\,d#1}
\newcommand{\dxdy}[2]{\frac{\partial #1}{\partial #2}}
\newcommand{\dxy}[2]{\frac{d #1}{d #2}}
\newcommand{\dydx}[1]{\frac{\partial #1}{\partial x}}
\newcommand{\dxdt}[1]{\frac{\partial #1}{\partial t}}
\newcommand{\dxdz}[1]{\frac{\partial #1}{\partial z}}
\newcommand{\dfdt}[1]{\frac{\partial}{\partial t} \fn{#1}}
\newcommand{\dfdz}[1]{\frac{\partial}{\partial z} \fn{#1}}
\newcommand{\ddt}[1]{\frac{\partial}{\partial t} #1}
\newcommand{\ddz}[1]{\frac{\partial}{\partial z} #1}
\newcommand{\dd}[2]{\frac{\partial}{\partial #1} #2}
\newcommand{\ddx}[1]{\frac{\partial}{\partial x} #1}
\newcommand{\ddy}[1]{\frac{\partial}{\partial y} #1}
\newcommand{\dxdyn}[3]{\frac{\partial ^{#3} #1 }{\partial #2 ^{#3}}}
\newcommand{\Dxdy}[2]{\frac{D #1}{D #2}}
\newcommand{\Dxy}[2]{\frac{D #1}{D #2}}
%
% Vector forms
%
\renewcommand{\vec}[1]{\mbox{$\stackrel{\longrightarrow}{#1}$}}
\renewcommand{\div}{\mbox{$\vec{\nabla} \cdot$}}
\newcommand{\grad}{\mbox{$\vec{\nabla}$}}
\newcommand{\bb}[1]{\bar{\bar{#1}}}
%
% Equation beginnings and endings
%
\newcommand{\bea}{\begin{eqnarray}}
\newcommand{\eea}{\end{eqnarray}}
\newcommand{\be}{\begin{equation}}
\newcommand{\ee}{\end{equation}}
\newcommand{\beas}{\begin{eqnarray*}}
\newcommand{\eeas}{\end{eqnarray*}}
\newcommand{\bdm}{\begin{displaymath}}
\newcommand{\edm}{\end{displaymath}}
%
% Equation punctuation
%
\newcommand{\pec}{\, ,}
\newcommand{\pep}{\, .} 
\newcommand{\pev}{\hspace{0.25in}}
%
% Equation labels and references, figure references, table references
%
\newcommand{\lequ}[1]{\label{eq:#1}}
\newcommand{\equ}[1]{Eq.~(\ref{eq:#1})}
\newcommand{\equs}[1]{Eqs.~(\ref{eq:#1})}
\newcommand{\requ}[1]{(\ref{eq:#1})}
\newcommand{\lfig}[1]{\label{fi:#1}}
\newcommand{\fig}[1]{Fig.~\ref{fi:#1}}
\newcommand{\figs}[1]{Figs.~\ref{fi:#1}}
\newcommand{\rfig}[1]{\ref{fi:#1}}
\newcommand{\lta}[1]{\label{ta:#1}}
\newcommand{\ta}[1]{Table~\ref{ta:#1}}
\newcommand{\rta}[1]{\ref{ta:#1}}
%
% Superscript and subscript in text
%
\newcommand{\supertext}[1]{\ensuremath{^{\textrm{#1}}}}
\newcommand{\subtext}[1]{\ensuremath{_{\textrm{#1}}}}
%
%
% List beginnings and endings
%
\newcommand{\bl}{\bss\begin{itemize}}
\newcommand{\el}{\vspace{-.5\baselineskip}\end{itemize}\ess}
\newcommand{\ben}{\bss\begin{enumerate}}
\newcommand{\een}{\vspace{-.5\baselineskip}\end{enumerate}\ess}
%
% Figure and table beginnings and endings
%
\newcommand{\bfg}{\begin{figure}}
\newcommand{\efg}{\end{figure}}
\newcommand{\bt}{\begin{table}}
\newcommand{\et}{\end{table}}
%
% Tabular and center beginnings and endings
%
\newcommand{\bc}{\begin{center}}
\newcommand{\ec}{\end{center}}
\newcommand{\btb}{\begin{center}\begin{tabular}}
\newcommand{\etb}{\end{tabular}\end{center}}
%
% Single space command
%
\newcommand{\bss}{\begin{singlespace}}
\newcommand{\ess}{\end{singlespace}}
%
%---New environment "arbspace". (modeled after singlespace environment
%                                in Doublespace.sty)
%   The baselinestretch only takes effect at a size change, so do one.
%
\def\arbspace#1{\def\baselinestretch{#1}\@normalsize}
\def\endarbspace{}
\newcommand{\bas}{\begin{arbspace}}
\newcommand{\eas}{\end{arbspace}}
%
% An explanation for a function
%
\newcommand{\explain}[1]{\mbox{\hspace{2em} #1}}
%
% Quick commands for symbols
%
\newcommand{\half}{\frac{1}{2}}
\newcommand{\third}{\frac{1}{3}}
\newcommand{\twothird}{\frac{2}{3}}
\newcommand{\fourth}{\frac{1}{4}}
\newcommand{\mdot}{\dot{m}}
%\newcommand{\ten}[1]{\times 10^{#1}\,}
\newcommand{\cL}{{\cal L}}
\newcommand{\cD}{{\cal D}}
\newcommand{\cF}{{\cal F}}
\newcommand{\cE}{{\cal E}}
\renewcommand{\Re}{\mbox{Re}}
\newcommand{\Ma}{\mbox{Ma}}
\newcommand{\mA}{\mathbf{A}}
\newcommand{\mB}{\mathbf{B}}
\newcommand{\mC}{\mathbf{C}}
\newcommand{\E}{\mathcal{E}}
\newcommand{\F}{\mathcal{F}}
\newcommand{\dt}{\Delta t}
\newcommand{\iL}{_{i,L}}
\newcommand{\iR}{_{i,R}}

%
% Inclusion of Graphics Data
%
%\input{psfig}
%\psfiginit
%
% More Quick Commands
%
\newcommand{\bi}{\begin{itemize}}
\newcommand{\ei}{\end{itemize}}
\newcommand{\dxi}{\Delta x_i}
\newcommand{\dyj}{\Delta y_j}
\newcommand{\ts}[1]{\textstyle #1}

\begin{document}

%===============================================================================
\section{Introduction}
%===============================================================================

In this work, a new IMEX scheme for solving the equations of radiation
hydrodynamics (RH) that is second-order accurate in both space and time is
presented and tested. A RH system combining a 1-D slab model
of compressible fluid dynamics with a grey radiation S$_2$ model is considered,
given by:
\begin{subequations}
\lequ{radhydro_system}
\be
\dxdy{\rho}{t}+\dxdy{}{x}\fn{\rho u} = 0 \pec
\lequ{cons_mass}
\ee 
\be
\dxdy{}{t}\fn{\rho u} + \dxdy{}{x}\fn{\rho u^2} + \dxdy{}{x}\fn{p}
  = \frac{\sigma_t}{c} \F_0 \pec
\lequ{cons_mom}
\ee
\be
\dxdy{E}{t} + \dxdy{}{x}\bracket{\fn{E+p}u}=-\sigma_a c \fn{aT^4 - \E}
  + \frac{\sigma_t u}{c} \F_0 \pec
\lequ{cons_energy}
\ee
\be
\frac{1}{c}\dxdy{\psi^+}{t} + \frac{1}{\sqrt{3}}\dxdy{\psi^+}{x}
  + \sigma_t \psi^+ = \frac{\sigma_s}{4\pi} c\E + \frac{\sigma_a}{4\pi} acT^4
  - \frac{\sigma_t u}{4\pi c} \F_0 + \frac{\sigma_t}{\sqrt{3}\pi}\E u
\pec
\lequ{intp}
\ee

\be
\frac{1}{c}\dxdy{\psi^-}{t} - \frac{1}{\sqrt{3}}\dxdy{\psi^-}{x}
  + \sigma_t \psi^- = \frac{\sigma_s}{4\pi} c\E + \frac{\sigma_a}{4\pi} acT^4
  - \frac{\sigma_t u}{4\pi c} \F_0 - \frac{\sigma_t}{\sqrt{3}\pi}\E u
\pec
\lequ{intm}
\ee
\end{subequations}
where $\rho$ is the density, $u$ is the velocity,
$E=\rho\fn{\frac{u^2}{2} + e}$ is the total material energy density,
$e$ is the specific internal energy density, $T$ is the material temperature,
$\E$ is the radiation energy density:
\be
\E = \frac{2\pi}{c}\fn{\psi^{+}+\psi^{-}} \pec
\lequ{Erad}
\ee
$\F$ is the radiation energy flux:
\be
\F = \frac{2\pi}{\sqrt{3}}\fn{\psi^{+}-\psi^{-}} \pec
\lequ{flux}
\ee
and $\F_0$ is an approximation to the comoving-frame flux,
\be
\lequ{F_nu_0}
\F_0 = \F-\frac{4}{3} \E u \pep
\ee
Note that multiplying Eqs.~\requ{intp} and \requ{intm} by $2\pi$ and summing
them gives the radiation energy equation:
\begin{subequations}
\be
\dxdy{\E}{t} + \dxdy{\F}{x} = \sigma_a c(aT^4 - \E) - \frac{\sigma_t u}{c}\F_0 \pec
\lequ{erad}
\ee
and multiplying \equ{intp} by $\frac{2\pi}{c\sqrt{3}}$ and \equ{intm} by
$-\frac{2\pi}{c\sqrt{3}}$ and sum them, we get the radiation momentum equation: 
\be
\frac{1}{c^2}\dxdy{\F}{t} + \frac{1}{3}\dxdy{\E}{x} = -\frac{\sigma_t}{c}\F_0 \pep
\ee
\end{subequations}

Equations \requ{cons_mass} through \requ{intm} are closed by assuming an ideal
equation of state (EOS):
\begin{subequations}
\be
p=\rho e (\gamma -1)
\lequ{pressure}
\pec
\ee
\be
T = \frac{e}{c_v} \pec
\lequ{matemp}
\ee
\end{subequations}
where $\gamma$ is the adiabatic index, and $c_v$ is the specific heat capacity.
However, the method presented is compatible with any valid EOS. 

%===============================================================================
\section{Spatial Discretization of the S$_2$ Equations}
%===============================================================================

Taking the zeroth angular moment of the S$_2$ equations, given by Equations
\requ{intp} and \requ{intm}, gives
\be
\frac{1}{c}\dxdt{\phi} + \dydx{\F} + \sigma_t\phi = \sigma_s\phi + Q_0 \pec
\lequ{zerothmoment}
\ee
where $\phi = c\E$ and the source $Q_0$ is
\be
Q_0 = \sigma_a acT^4 - \sigma_t\frac{u}{c}\F_0 \pep
\lequ{Q0}
\ee
Taking the first angular moment of the S$_2$ equations gives
\be
\frac{1}{c}\dxdt{\F} + \dydx{\phi} + \sigma_t\F = Q_1 \pec
\lequ{firstmoment}
\ee
where $Q_1$ is
\be
Q_1 = \sigma_a acT^4 - \sigma_t\frac{u}{c}\F_0 \pep
\lequ{Q1}
\ee
Rewriting the S$_2$ equations and employing the definitions given by Equations
\requ{Q0} and \requ{Q1} gives
\be
\frac{1}{c}\dxdt{\psi^\pm} + \mu^\pm\dydx{\psi^\pm} + \sigma_t\psi^\pm
  = \frac{\sigma_s}{4\pi}\phi + \frac{1}{4\pi}Q_0 + \frac{3\mu^\pm}{4\pi}Q_1 \pec
\lequ{S2Q}
\ee
where $\mu^\pm=\pm\frac{1}{\sqrt{3}}$.
A lumped linear discontinuous spatial discretization is employed, so the
numerical unknowns are the left and right values $\psi_{i,L}^\pm$ and
$\psi_{i,R}^\pm$ for each cell $i$. The spatially
discretized equations result from integrating each half cell
$(x_i-\frac{\Delta x_i}{2},x_i)$ and $(x_i,x_i+\frac{\Delta x_i}{2})$,
where $x_i$ is the cell center, and $\Delta x_i$ is the cell width.
Integrating the first half cell gives
\begin{multline}
\frac{1}{c}\dxdt{\psi^\pm_{i,L}}\frac{\Delta x_i}{2}
  + \mu^\pm\fn{\psi_i^\pm-\psi_{i-\half}^\pm}
  + \sigma_{t,i,L}\psi_{i,L}^\pm\frac{\Delta x_i}{2} \\
  = \frac{\sigma_{s,i,L}}{4\pi}\phi_{i,L}\frac{\Delta x_i}{2}
  + \frac{1}{4\pi}Q_{0,i,L}\frac{\Delta x_i}{2}
  + \frac{3\mu^\pm}{4\pi}Q_{1,i,L}\frac{\Delta x_i}{2} \pec
\lequ{S2left}
\end{multline}
where $\psi_i^\pm=\half\fn{\psi_{i,L}^\pm + \psi_{i,R}^\pm}$,
$\psi_{i+\half}^+=\psi_{i,R}^+$, and $\psi_{i+\half}^-=\psi_{i+1,L}^-$.
Similarily, for the second half cell,
\begin{multline}
\frac{1}{c}\dxdt{\psi^\pm_{i,R}}\frac{\Delta x_i}{2}
  + \mu^\pm\fn{\psi_{i+\half}^\pm-\psi_i^\pm}
  + \sigma_{t,i,R}\psi_{i,R}^\pm\frac{\Delta x_i}{2} \\
  = \frac{\sigma_{s,i,R}}{4\pi}\phi_{i,R}\frac{\Delta x_i}{2}
  + \frac{1}{4\pi}Q_{0,i,R}\frac{\Delta x_i}{2}
  + \frac{3\mu^\pm}{4\pi}Q_{1,i,R}\frac{\Delta x_i}{2} \pep
\lequ{S2right}
\end{multline}

\end{document}
