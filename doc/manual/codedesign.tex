\documentclass[preprint,12pt]{elsarticle}
\usepackage{amssymb}
\usepackage{hyperref}
\usepackage{amsmath}
\usepackage{color}
\usepackage[procnames]{listings}
% Mark URL's
\newcommand{\URL}[1]{{\textcolor{blue}{#1}}}

%\newcommand{\elfit}{\ensuremath{\operatorname{Im}(-1/\epsilon(\vq,\omega)}}
%\msection{}-->section commands
%\tradem{}  -->add TM subscript to entry
%\ucatm{}   -->add trademark footnote about entry
%
% Ways of grouping things
%
\newcommand{\bracket}[1]{\left[ #1 \right]}
\newcommand{\bracet}[1]{\left\{ #1 \right\}}
\newcommand{\fn}[1]{\left( #1 \right)}
\newcommand{\ave}[1]{\left\langle #1 \right\rangle}
\newcommand{\norm}[1]{\Arrowvert #1 \Arrowvert}
\newcommand{\abs}[1]{\arrowvert #1 \arrowvert}

%
% Derivative forms
%
\newcommand{\dx}[1]{\,d#1}
\newcommand{\dxdy}[2]{\frac{\partial #1}{\partial #2}}
\newcommand{\dxy}[2]{\frac{d #1}{d #2}}
\newcommand{\dydx}[1]{\frac{\partial #1}{\partial x}}
\newcommand{\dxdt}[1]{\frac{\partial #1}{\partial t}}
\newcommand{\dxdz}[1]{\frac{\partial #1}{\partial z}}
\newcommand{\dfdt}[1]{\frac{\partial}{\partial t} \fn{#1}}
\newcommand{\dfdz}[1]{\frac{\partial}{\partial z} \fn{#1}}
\newcommand{\ddt}[1]{\frac{\partial}{\partial t} #1}
\newcommand{\ddz}[1]{\frac{\partial}{\partial z} #1}
\newcommand{\dd}[2]{\frac{\partial}{\partial #1} #2}
\newcommand{\ddx}[1]{\frac{\partial}{\partial x} #1}
\newcommand{\ddy}[1]{\frac{\partial}{\partial y} #1}
\newcommand{\dxdyn}[3]{\frac{\partial ^{#3} #1 }{\partial #2 ^{#3}}}
\newcommand{\Dxdy}[2]{\frac{D #1}{D #2}}
\newcommand{\Dxy}[2]{\frac{D #1}{D #2}}
%
% Vector forms
%
\renewcommand{\vec}[1]{\mbox{$\stackrel{\longrightarrow}{#1}$}}
\renewcommand{\div}{\mbox{$\vec{\nabla} \cdot$}}
\newcommand{\grad}{\mbox{$\vec{\nabla}$}}
\newcommand{\bb}[1]{\bar{\bar{#1}}}
%
% Equation beginnings and endings
%
\newcommand{\bea}{\begin{eqnarray}}
\newcommand{\eea}{\end{eqnarray}}
\newcommand{\be}{\begin{equation}}
\newcommand{\ee}{\end{equation}}
\newcommand{\beas}{\begin{eqnarray*}}
\newcommand{\eeas}{\end{eqnarray*}}
\newcommand{\bdm}{\begin{displaymath}}
\newcommand{\edm}{\end{displaymath}}
%
% Equation punctuation
%
\newcommand{\pec}{\, ,}
\newcommand{\pep}{\, .} 
\newcommand{\pev}{\hspace{0.25in}}
%
% Equation labels and references, figure references, table references
%
\newcommand{\lequ}[1]{\label{eq:#1}}
\newcommand{\equ}[1]{Eq.~(\ref{eq:#1})}
\newcommand{\equs}[1]{Eqs.~(\ref{eq:#1})}
\newcommand{\requ}[1]{(\ref{eq:#1})}
\newcommand{\lfig}[1]{\label{fi:#1}}
\newcommand{\fig}[1]{Fig.~\ref{fi:#1}}
\newcommand{\figs}[1]{Figs.~\ref{fi:#1}}
\newcommand{\rfig}[1]{\ref{fi:#1}}
\newcommand{\lta}[1]{\label{ta:#1}}
\newcommand{\ta}[1]{Table~\ref{ta:#1}}
\newcommand{\rta}[1]{\ref{ta:#1}}
%
% Superscript and subscript in text
%
\newcommand{\supertext}[1]{\ensuremath{^{\textrm{#1}}}}
\newcommand{\subtext}[1]{\ensuremath{_{\textrm{#1}}}}
%
%
% List beginnings and endings
%
\newcommand{\bl}{\bss\begin{itemize}}
\newcommand{\el}{\vspace{-.5\baselineskip}\end{itemize}\ess}
\newcommand{\ben}{\bss\begin{enumerate}}
\newcommand{\een}{\vspace{-.5\baselineskip}\end{enumerate}\ess}
%
% Figure and table beginnings and endings
%
\newcommand{\bfg}{\begin{figure}}
\newcommand{\efg}{\end{figure}}
\newcommand{\bt}{\begin{table}}
\newcommand{\et}{\end{table}}
%
% Tabular and center beginnings and endings
%
\newcommand{\bc}{\begin{center}}
\newcommand{\ec}{\end{center}}
\newcommand{\btb}{\begin{center}\begin{tabular}}
\newcommand{\etb}{\end{tabular}\end{center}}
%
% Single space command
%
\newcommand{\bss}{\begin{singlespace}}
\newcommand{\ess}{\end{singlespace}}
%
%---New environment "arbspace". (modeled after singlespace environment
%                                in Doublespace.sty)
%   The baselinestretch only takes effect at a size change, so do one.
%
\def\arbspace#1{\def\baselinestretch{#1}\@normalsize}
\def\endarbspace{}
\newcommand{\bas}{\begin{arbspace}}
\newcommand{\eas}{\end{arbspace}}
%
% An explanation for a function
%
\newcommand{\explain}[1]{\mbox{\hspace{2em} #1}}
%
% Quick commands for symbols
%
\newcommand{\half}{\frac{1}{2}}
\newcommand{\third}{\frac{1}{3}}
\newcommand{\twothird}{\frac{2}{3}}
\newcommand{\fourth}{\frac{1}{4}}
\newcommand{\mdot}{\dot{m}}
%\newcommand{\ten}[1]{\times 10^{#1}\,}
\newcommand{\cL}{{\cal L}}
\newcommand{\cD}{{\cal D}}
\newcommand{\cF}{{\cal F}}
\newcommand{\cE}{{\cal E}}
\renewcommand{\Re}{\mbox{Re}}
\newcommand{\Ma}{\mbox{Ma}}
\newcommand{\mA}{\mathbf{A}}
\newcommand{\mB}{\mathbf{B}}
\newcommand{\mC}{\mathbf{C}}
\newcommand{\E}{\mathcal{E}}
\newcommand{\F}{\mathcal{F}}
\newcommand{\dt}{\Delta t}
\newcommand{\iL}{_{i,L}}
\newcommand{\iR}{_{i,R}}

%
% Inclusion of Graphics Data
%
%\input{psfig}
%\psfiginit
%
% More Quick Commands
%
\newcommand{\bi}{\begin{itemize}}
\newcommand{\ei}{\end{itemize}}
\newcommand{\dxi}{\Delta x_i}
\newcommand{\dyj}{\Delta y_j}
\newcommand{\ts}[1]{\textstyle #1}



%%%%%%%%%%%%%%%%%%%%%%%%%%%%%%%%%%%%%%%%%%%%%%%%%%%%%
% SIMONS MACROS 
%%%%%%%%%%%%%%%%%%%%%%%%%%%%%%%%%%%%%%%%%%%%%%%%%%%%%
\newcommand{\deriv}[2]{\frac{\mathrm{d} #1}{\mathrm{d} #2}}
\newcommand{\pderiv}[2]{\frac{\partial #1}{\partial #2}}
\newcommand{\bx}{\mathbf{X}}
\newcommand{\ba}{\mathbf{A}}
\newcommand{\by}{\mathbf{Y}}
\newcommand{\bj}{\mathbf{J}}
\newcommand{\bs}{\mathbf{s}}
\newcommand{\B}[1]{\ensuremath{\mathbf{#1}}}
\newcommand{\Dt}{\Delta t}
\renewcommand{\d}{\mathrm{d}}
\newcommand{\mom}[1]{\langle #1 \rangle}
\newcommand{\cur}[1]{\left\{ #1 \right\}}
\newcommand{\xl}{{x_{i-1/2}}}
\newcommand{\xr}{{x_{i+1/2}}}
\newcommand{\il}{{i-1/2}}
\newcommand{\ir}{{i+1/2}}

\newcommand{\ER}{\mathcal{E}}
\newcommand{\FR}{\mathcal{F}}

%Code input stuff for python
\definecolor{keywords}{RGB}{255,0,90}
\definecolor{comments}{RGB}{0,0,113}
\definecolor{red}{RGB}{160,0,0}
 
\lstset{language=Python, 
        basicstyle=\ttfamily\small, 
        keywordstyle=\color{keywords},
        commentstyle=\color{comments},
        stringstyle=\color{red},
        showstringspaces=false,
        identifierstyle=\color{blue},
        procnamekeys={def,class}}

\begin{document}

\section{Code Design}

\subsection{A Couple Warnings about Python}
Variables in python are only names: they are \emph{not} a pointer
or a value. This can lead to unusal behavior compared to C or Fortran.
Nonmutable objects (doubles, ints, etc.) will behave as expected. Mutable
objects (strings, lists, arrays, classes, i.e., things that have a state that can be
changed) in assignment (= operator) by default do not
construct a new object. Instead, they just assign the new name to the same object.
This is done to save memory when dealing with large objects.   Thus, 
\begin{lstlisting}
#Immutable
a = 5.
b = a
b += 2.
print a            #will yield 5.
print b            #will yield 7.     

#Mutable
a = [1,2,3]        #There is a list [1,2,3], we name it 'a'
b = a              #There is a list [1,2,3], it also has a name 'b'
b[0] = 5 
print a            #will yield [5,2,3]
\end{lstlisting}
This means you can accidentally change an array without meaning
to. Passes of mutable
objects to functions demonstrate the same behavior.  But b is \emph{not} a pointer to a, it just happens to own the same data
currently. Thus,
\begin{lstlisting}
a = [1,2,3]
b = a
b = [4]     #assign b to name a new list [4]
b[0] = 5
print a     #will yield [1,2,3]
\end{lstlisting}
To copy the contents of a to b, if it is a class, use the deepcopy command.  If it is
something simple, like a list, just use the constructor, i.e., \verb{b = list(a){.
    Numpy arrays have a .copy() method that works as well.

You need to be careful when working with classes in python. Python is ``duck typed''
\url{http://en.wikipedia.org/wiki/Duck_typing}. What this means in short is that
python does not care what you pass it, as long as it has the methods you asked it
for.  If you try to assign to a class member (e.g., a.member=5) and that class does
not have that member, it will make it have it for you.  Also, be sure to construct
\emph{instances} and not accidentally modify the \emph{class}\dots because it will
let you do that:
\begin{lstlisting}
a = MyClass(args)   #some class that takes arugments
a.name = 5       
MyClass.name = 5    #This is not the same as a.name
                    #It creates a new class variable called 'name' that now all
                    instances will have. 
\end{lstlisting}
Our naming conventions should take care of this problem. In general, define all
variables in the constructor to avoid similar issues:
\begin{lstlisting}
class myGoodClass:

    #define constructor, all member functions are passed 'self'
    def __init__(self, a,b,arr=[]): #default for arr is empty list
        self.a = a
        self.b = b
        self.arr = arr

    BAD_VAR = [] #DO NOT DEFINE VARIABLES AT CLASS SCOPE!!!!
    
\end{lstlisting}



\subsection{Naming Convention}
\begin{itemize}
    \item Functions: nameOfFunction (first letter lower case)
    \item Variables: name\_sep\_by\_underscores (upper case only for special
    variables, e.g., \verb}rho_F} for rho Flux)
    \item Class Names: NameOfClass (first letter capital)
    \item Member Variables: same as variable naming convention
    \item Function Definitions: in general functions can be done with pass by order
        (e.g., func(a,b,c)).
        Python is also very effective at keyword arguments, which have default values (e.g.,
        func(a,b,mesh=my\_mesh). I recommend using keyword arguments when possible if
        there is a large number of variables of similar form (e.g., they are all long
        lists) being passed to a function.
    \item Counters and lengths: all vars starting with \verb{n_{ are integer sizes
    \item Use 85 character line widths
\end{itemize}

\subsection{Basic Data Structures}

\begin{enumerate}
    \item S$_2$ fluxes: Vectors, DOF, and equations corresponding to S$_2$ equations over an element will
        be ordered $\psi_L^+,\psi_R^+,\psi_L^-,\psi_R^-$, e.g., the
        source terms for the equations over an element will be passed into the
        radiation solver corresponding to the equations in the above order.  This will result in a 7-banded matrix (not the ideal of 5, but
        good enough for this).  Scipy has some sparse matrix options, but I don't
        think we will probably need it. numpy.linalg.solve is easier to use.
    \item Arrays: We will be using numpy (np) arrays for all matrices and vectors.  Have
        to be careful with syntax on commands, e.g., $A*B$ for two 2D numpy arrays
        does not do a matrix multiply, for that you want \verb{np.dot(A,B){.  Numpy arrays are indexed like c arrays. For a 2D
            array a[0][2] gives value of  first row, third column. I find the syntax
            for creating np arrays to be a pain sometimes, so I often just use lists
            of lists and then convert them to a np array. This is slow, but our run
            times should be pretty fast anyways.  Also
            note we are not using scipy.matrix in general, unless we need it temporarily
            for a  special command. If you do, note that indexing for the matrix
            classes is done as a[0,2] instead.
    \item Other Data: If anything is accessed by a string use a dictionary, otherwise
        just use lists or numpy arrays. Numpy arrays and dictionaries can take lists
        of any type, indexed by any type
    \item Global constants - $c$, $a$, etc. Stored in a dictionary that is globally
        accessible. Stored in it's own module file.
    \item Spatial Mesh
    \begin{itemize}
        \item List of edge coordinates, element widths, element centers. 
        \item Basic accessor functions to get values for a particular element
    \end{itemize}   
    \item Hydro State Variable
    \begin{itemize}
        \item Class object
        \item Contains the full state of system at a point, i.e., $\rho, E, e, u, p$
        \item Has methods for EOS updates to pressure etc.
        \item note, \emph{all object members are public in python}, so dont really need accessor
            functions
        \item The state update functions may need to be modified in order to make the radiation
            solver work correctly since it updates internal energy.
    \end{itemize}
    \item Radiation State
    \begin{itemize}
        \item Fundamental unknowns are $\psi^+$ and $\psi^-$ at a point, but there is
            no need to store these in a class because only the radiation solver
            cares.
        \item We can store $\ER$ and $\FR$ in a class that recieves $\psi^+$ and
            $\psi^-$ in constructor, if necessary.
    \end{itemize}
    \item Structure of variables on the mesh: 
        \begin{itemize}
            \item Radiation variables are at points left and right,
        there is no need for an average and a slope in general. 
            \item Hydro variables will
        exist as averages and slopes. L and R values are reconstructed as needed, but
        all functions are expected to modify (or return) the averages and slopes. 
            \item We can just store a list of the states for each element, indexed by
                number, and pass these lists around to solvers etc.  This will decouple our code better than having the mesh store
        variables, and will be easier with all the different time steps. 
    \end{itemize}
    \item CrossSection Class:
    \begin{itemize}
        \item Class that contains $\sigma_s$, $\sigma_a$, $\sigma_t$
        \item Use external functions or classes to generate the Newton specific cross
            sections, and $\sigma_t + 1/c \Delta t$.  Then pass a list of cross
            sections objects, for all elements, to solvers.
    \end{itemize}
    \item SourceHandler Class:
    \begin{itemize}
        \item This class is a base class for implementing derived class sources in
            building the radiation system. 
        \item Each term that goes on the RHS of the equation (anything other than
            implicit streaming, interaction terms) is responsible for implementing
            how to evaluate their $Q$ contribution for BE, CN, and BDF2 time
            differencing schemes.
        \item These are all located in sourceHandler source file
        \item See source code for more details.
    \end{itemize}

\end{enumerate}

\subsection{Method Classes}

There is a lot of similarities between steps that we can utilize. Our main solvers
will be predictor and corrector hydro solvers and an S$_2$ solver with a fixed $Q_0$
and $Q_1$. The rest of it boils down to setting the correct coefficients and building
the sources.  

\begin{enumerate}
    \item Hydro Predictor
    \begin{itemize}
        \item \textbf{What is fixed?}: Always a CN step
        \item \textbf{What changes?}: $\Dt$, 
        \item Interface
        \begin{itemize}
        \item \textbf{Inputs:} $\Dt$, old HydroStates averages
        \item \textbf{Output}: new HydroStates, averages and slopes
        \end{itemize}
    \end{itemize}
    \item Hydro Corrector
    \begin{itemize}
        \item The hydro solvers are already implemented, but will need to have the predictor and corrector seperated
    
     \end{itemize}
     \item Method to convert from avg slope to $L$ and $R$, for a single element

    \item Hydro Predictor
    \begin{itemize}
        \item \textbf{What is fixed?}: Always a CN step
        \item \textbf{What changes?}: $\Dt$, 
        \item Interface
        \begin{itemize}
        \item \textbf{Inputs:} $\Dt$, old HydroStates averages
        \item \textbf{Output}: new HydroStates, averages and slopes
        \end{itemize}
    \end{itemize}

    
\end{enumerate}

\end{document}
